\chapter{Studi Literatur}

Pada bab ini akan mendeskripsikan kajian literatur yang terkait dengan persoalan tugas akhir. Studi literatur ini akan dijadikan dasar dalam melakukan penyelesaian persoalan.

\section{\textit{Spreadsheet}}
Secara harafiah, \textit{spreadsheet} adalah suatu perangkat lunak yang dapat melakukan kalkulasi terhadap angka serta mengorganisir informasi yang ada di dalamnya berdasarkan kolom dan baris \parencite{meriamwebster-spreadsheet}. Konsep dasar pada aplikasi \textit{spreadsheet} modern adalah sebuah aplikasi yang berupa sekumpulan sel terdiri dari baris dan kolom yang disebut \textit{sheet} yang dapat digambarkan sebagai matriks yang besar \parencite{Ronen1989}.

Sel-sel pada \textit{spreadsheet} dapat diisi data berupa data mentah maupun formula. Data mentah dapat berupa angka, teks, tanggal, dan nilai mata uang. Formula merupakan perintah yang dapat dimengerti komputer untuk menghitung dan memanipulasi data pada sel. Data hasil pengolahan dan masukan pada \textit{spreadsheet} ditampilkan dalam bentuk sel yang namanya terdiri dari nama kolom dan nilai baris (Contoh: A1 untuk kolom pertama dan baris pertama). Selain itu, sel tersebut juga dapat memiliki \textit{properties} berupa \textit{value} yang diisikan, format sel, serta format data yang digunakan.

\section{Penggunaan \textit{Spreadsheet}}
\textit{Spreadsheet} dapat digunakan untuk melakukan kalkulasi terhadap suatu rumus atau formula yang sulit jika dikalkulasikan dengan cara manual. Selain itu, \textit{spreadsheet} dapat juga digunakan untuk melakukan ramalan terhadap suatu perubahan variabel masukan. Pada perkembangannya, \textit{spreadsheet} memiliki fitur-fitur tambahan seperti visualisasi data dan ekstraksi data penting dari kumpulan data yang ada.

Penelitian tentang penggunaan \textit{spreadsheet} pada bisnis pernah dilakukan sebelumnya pada tahun 2014. Subjek yang diteliti adalah akuntan manajemen \parencite{Bradbard2014}. Pada penelitian tersebut, didapatkan gambaran umum mengenai penggunaan \textit{spreadsheet} secara umum. Menurut hasil penelitian tersebut beberapa fitur yang sering digunakan oleh pengguna \textit{spreadsheet} secara terurut dari yang paling sering digunakan adalah sebagai berikut,

\begin{enumerate}
    \item Menghitung fungsi matematika dasar (tambah, kurang, kali, bagi, dan lainnya)
    \item Mengelola \textit{worksheet} dan \textit{workbook} (menambahkan, menghapus, merubah nama, dan lainnya)
    \item Melakukan perubahan format dasar (menebalkan, memberi garis bawah, format angkat, dan lainnya)
    \item Melakukan pengurutan data, penghitungan subtotal, serta meringkas data
    \item Menggunakan fitur \textit{cell addressing} baik absolut maupun relatif
    \item Penggunaan fungsi kondisi (IF, COUNTIF), fungsi logika (AND, OR), fungsi pencarian (VLOOKUP, HLOOKUP), menautkan \textit{workbook} lain, serta fungsi pembulatan (ROUND, CEILING, FLOOR)
\end{enumerate}

Penggunaan \textit{spreadsheet} sangat bergantung kepada domain bisnis atau organisasi yang menggunakan. Pada bisnis yang berorientasi komersial, \textit{spreadsheet} dapat digunakan sebagai alat bantu perhitungan laba, pengeluaran, investasi, dan pajak. Pada organisasi-organisasi non komersial, \textit{spreadsheet} dapat digunakan sebagai salah satu bentuk basis data yang menangani penyimpanan, pengelolaan, dan pengumpulan data yang mudah dan cepat.

\section{Kesalahan dalam Penggunaan \textit{Spreadsheet}}
Penelitian telah dilakukan oleh Panko \parencite{Panko1998} untuk mengetahui banyaknya kesalahan yang terjadi pada pengembangan \textit{spreadsheet} terutama pada sektor bisnis. Dari penelitian ini, didapatkan bahwa 20\% hingga 40\% \textit{spreadsheet} mengandung kesalahan. Pada kasus tertentu, bahkan ditemukan 90\% \textit{spreadsheet} yang diteliti memiliki kesalahan \parencite{Journal1996}. 

Penelitian yang dilakukan oleh Panko juga menemukan 88\% dari 113 \textit{spreadsheet} yang diaudit melalui 7 lebih studi yang diteliti. Beberapa hasil yang telah di rangkum oleh penelitian tersebut mengunakan \textit{spreadsheet} yang digunakan di dunia nyata dapat dilihat pada Tabel \ref{StudiKesalahan}.
  \begin{longtable}{ | p{3cm} | r | R{2cm} | r | r | r | l | }
    \caption{Studi terhadap Kesalahan pada \textit{Spreadsheet}}
    \label{StudiKesalahan}\\ \hline
    \centering\bfseries{Pembuat} & \bfseries{Tahun} & \centering\bfseries{Jumlah yang Diaudit} & \bfseries{Rata-rata Sel} & \bfseries{Persentase Error} \\ \hline
    \endfirsthead
    \hline
    \centering\bfseries{Pembuat} & \bfseries{Tahun} & \centering\bfseries{Jumlah yang Diaudit} & \bfseries{Rata-rata Sel} & \bfseries{Persentase Error} \\ \hline
    \endhead
    Davies \& Ikin & 1987 & 19 & - & 21\% \\ \hline
    Cragg \& King & 1992 & 20 & 50 - 10000 & 25\% \\ \hline
    Butler & 1992 & 273 & - & 11\% \\ \hline
    Dent & 1994 & Tidak diketahui & - & 30\% \\ \hline
    Hicks & 1995 & 1 & 3856 & 100\% \\ \hline
    Coopers \& Lybrand & 1997 & 23 & 150+ & 91\% \\ \hline
    KPMG & 1998 & 22 & - & 91\% \\ \hline
    Lukasic & 1998 & 2 & 2270 - 7027 & 100\% \\ \hline
    Butler & 2000 & 7 & - & 86\% \\ \hline
    Clermont, Hanin, \& Mittermeier & 2002 & 3 & - & 100\% \\ \hline
    Lawrence and Lee & 2004 & 30 & 2182 & 100\% \\ \hline
    Powell, Lawson, and Baker & 2007 & 25 & - & 64\% \\ \hline
    Powell, Baker \& Lawson & 2007 & 50 & - & 86\% \\ \hline
  \end{longtable}
Dari kumpulan data diatas, dapat dilihat bahwa didalam pembentukan \textit{spreadsheet} pada bidang bisnis, tidak mungkin terlepas dari kesalahan. Dengan tingginya tingkat kesalahan ini, bisnis dapat mengalami kerugian secara material maupun moral yang cukup besar \parencite{EUSPRIGHorrorStories}. Hal ini mengindikasikan bahwa tingginya tingkat kesalahan harus dapat diselesaikan agar tidak terjadi kerugian di dalam penggunaan \textit{spreadsheet} terutama dalam bisnis.

\section{Tipe Kesalahan dalam Penggunaan \textit{Spreadsheet}}
Tingkat fleksibilitas \textit{spreadsheet} yang tinggi memberikan keleluasaan kepada penggunanya untuk melakukan banyak manipulasi dan pengelolaan data. Tingginya fleksibilitas ini dapat berakibat mudahnya \textit{human error} terjadi pada saat penggunaan \textit{spreadsheet} yang menyebabkan terjadinya kesalahan-kesalahan pada data. Tipe-tipe kesalahan pada \textit{spreadsheet} dapat dibagi menjadi dua jenis tipe kesalahan yakni kesalahan kuantitatif, dan kesalahan kualitatif \parencite{Panko1998}. 

    \subsection{Kesalahan Kualitatif}
    Kesalahan kualitatif merupakan kesalahan yang berhubungan dengan kualitas \textit{spreadsheet} tersebut. Beberapa kesalahan yang dapat diklasifikasikan sebagai kesalahan kualitatif adalah \parencite{Powell2009}:

    \begin{enumerate}
        \item Melakukan \textit{hard-code} pada suatu angka di dalam formula
        \item Menggunakan formula yang panjang dalam perhitungan
        \item Susunan data yang tidak direncanakan dengan baik
        \item Tidak adanya dokumentasi mengenai \textit{spreadsheet} yang dibuat
    \end{enumerate}

    Kesalahan ini tidak langsung mengakibatkan nilai hasil keluaran yang salah namun menurunkan kualitas dari \textit{spreadsheet} tersebut \parencite{Rajalingham2001}. Selain itu, kesalahan kualitatif dapat menyebabkan kesalahan kuantitatif terutama pada saat penggunaan fungsi analisis \textit{what-if} pada \textit{spreadsheet} \parencite{Panko1998}.

    \subsection{Kesalahan Kuantitatif}
    Kesalahan ini mengakibatkan \textit{spreadsheet} mengeluarkan hasil dan nilai yang salah didalam operasi perhitungannya. Kesalahan kuantitatif dapat dibagi menjadi tiga tipe kesalahan yakni \parencite{Panko1998}:

    \begin{enumerate}
        \item Kesalahan mekanikal (\textit{mechanical error}) yang biasanya terjadi akibat kesalahan pengetikan angka atau rujukan sel yang salah pada suatu formula
        \item Kesalahan logika (\textit{logical error}) yang terjadi pada pembuatan formula yang salah atau penggunaan fungsi yang tidak tepat
        \item Kesalahan akibat kelalaian pada interpretasi situasi atau spesifikasi yang diberikan sehingga \textit{spreadsheet} yang dihasilkan tidak sesuai dengan domain permasalahan yang ada \parencite{Powell2009} (\textit{ommision error})
    \end{enumerate}

\section{Penanganan Kesalahan pada \textit{Spreadsheet}}
Berdasarkan penelitian yang dilakukan oleh Panko \parencite{Panko1998}, dijabarkan beberapa metode untuk menangani dan mengurangi kesalahan yang sering terjadi. Beberapa metode yang dapat digunakan yakni:

    \begin{enumerate}
        \item Membangun \textit{preliminary design} sebelum pembuatan \textit{spreadsheet} agar terdapat perencanaan yang baik di dalam pembangunan data di dalam \textit{spreadsheet}
        \item Melakukan proteksi terhadap sel yang tidak boleh diubah.
        \item Melakukan pengecekan terhadap semua rumus dan formula yang dimasukan bahkan hingga rumus yang cukup sederhana dengan cara melakukan pengecekan manual.
        \item Membuat dokumentasi untuk \textit{spreadsheet} yang dibuat.
        \item Tidak menekan pembuat \textit{spreadsheet} terhadap kesalahan yang dibuat dengan memberikan hukuman. Kesalahan yang terjadi pada \textit{spreadsheet} umumnya masih berada pada batas normal \textit{human error} sehingga memberikan hukuman akan membuat rasa takut dalam melaporkan kesalahan.
        \item Melakukan inspeksi terhadap formula, rumus, dan kode yang dibuat baik oleh individual maupun secara berkelompok.
    \end{enumerate}

\section{Teknologi \textit{Spreadsheet}}
Perkembangan teknologi \textit{spreadsheet} digital modern dimulai pada tahun 1978, saat Bricklin mengembangkan \textit{working prototype} dari konsep dasar \textit{spreadsheet} menggunakan Integer BASIC. Pada tahun yang sama, Frankston dan Fylstra bergabung dan membentuk sebuah perangkat lunak bernama VisiCalc (Visible Calculator) yang merupakan sebuah perangkat lunak \textit{spreadsheet} pertama yang bekerja dengan baik dan sukses dipasaran. Setelah keberhasilan VisiCalc, mulai muncul aplikasi serupa yang semakin baik salah satunya adalah Lotus. Dengan berkembangnya daya komputasi dan munculnya konsep \textit{graphical user interface}, Microsoft mengembangkan Microsoft Excel yang merupakan \textit{spreadsheet} pertama yang menggunakan antarmuka grafis dan menggunakan \textit{mouse} sebagai alat kontrol \parencite{power2004brief}.

Saat ini, perangkat lunak berupa \textit{spreadsheet} sangat banyak variasi dan tipenya. Perangkat lunak \textit{spreadsheet} ini dapat dibagi berdasarkan konektivitasnya yakni \textit{offline spreadsheet} dan \textit{online spreadsheet}. Selain itu, perangkat lunak \textit{spreadsheet} dapat juga dibagi berdasarkan keterbukaan dari \textit{source code} yakni \textit{open source} dan \textit{closed source}. Dari pembagian tersebut, beberapa perangkat lunak yang dapat dijadikan contoh untuk memenuhi ciri tersebut adalah; Microsoft Excel (\textit{Offline Spreadsheet}), Open Office (\textit{Offline and Open Source Spreadsheet}), EtherCalc (\textit{Online and Open Source Spreadsheet}), Google Sheet (\textit{Online Spreadsheet}). Bagian ini akan membahas masing-masing perangkat lunak tersebut secara umum.

    \subsection{Microsoft Excel}
    Microsoft Excel adalah perangkat lunak yang dikembangkan oleh Microsoft yang menyediakan fitur dasar dari \textit{spreadsheet} serta dengan fitur-fitur lainnya yang selalu ditambahkan pada setiap iterasi pengembangan Excel. Microsoft Excel dapat dimiliki oleh pengguna melalui pembelian paket Microsoft Office yang berisikan produk esensial Microsoft lainnya \parencite{MSExcelProduct}. Hingga Excel 2010, perangkat lunak ini memiliki beberapa keterbatasan seperti yang dapat dilihat pada Tabel \ref{KeterbatasanExcel}.

    \begin{table}[htb]
        \caption{Keterbatasan pada Microsoft Excel 2010 hingga versi terbaru}
        \label{KeterbatasanExcel}
        \begin{center}
            \begin{tabular}{ | l | c | }
                \hline
                Fitur & Batas \\ \hline
                Ukuran \textit{worksheet} & 1.048.576 baris dan 16.384 kolom \\
                Panjang kolom & 255 karakter \\
                Tinggi baris & 409 poin \\
                Jumlah format atau \textit{style} sel unik & 64.000 \\
                \textit{Hyperlink} pada \textit{worksheet} & 66.530 \textit{hyperlink} \\
                Tingkat presisi angka & 15 digit \\
                Panjang formula & 8.192 karakter \\
                \hline
            \end{tabular}
        \end{center}
    \end{table}

    Sejak Microsoft Excel 2007, Microsoft menggunakan format Office Open XML (OOXML) sebagai format penyimpanan \parencite{MSExcelSupport}. Office Open XML dikembangkan oleh Microsoft mulai dari tahun 2000 dengan diimplementasinya dukungan XML pada Microsoft Office 2000. Pada awal penggunaan aplikasi \textit{office}, terdapat permasalahan \textit{data interoperability} antar mesin dan sulitnya manipulasi data. Office Open XML diharapkan dapat menyelesaikan permasalahan ini dengan membentuk standar yang dapat diimplementasi berbagai aplikasi \textit{office} \parencite{OOXMLFormat}.

    Fitur-fitur yang ada pada Microsoft Excel terdiri dari fitur dasar yang tersedia pada \textit{spreadsheet} pada umumnya seperti perhitungan matematika, pengurutan dan penyaringan data, serta visualisasi data. Selain fitur dasar tersebut, terdapat pula fitur seperti \textit{import} dan \textit{export} data dari basis data. Dengan fitur ini, pengguna dapat mengambil data dari basis data dan menampilkannya dalam bentuk tabel sederhana serta memasukannya kembali ke basis data. Untuk mempermudah ekstraksi informasi, terdapat juga fitur \textit{pivot table} yang dapat mengolah banyak data menjadi informasi yang diinginkan \parencite{MSExcelSupport}. 

    \subsection{LibreOffice}


    \subsection{EtherCalc}


    \subsection{Google Sheet}


\section{\textit{Data Governance}}


\section{Metadata}


\section{Studi Terkait / Penelitian Terkait}
\blindtext
