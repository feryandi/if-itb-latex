\chapter{Rancangan, Implementasi, dan Pengujian}

\section{Perancangan Perangkat Lunak}
Subbab perancangan perangkat lunak menjelaskan deskripsi aplikasi, analisis kebutuhan fungsional dan non-fungsional, desain perangkat lunak, serta interaksinya.

	\subsection{Deskripsi Umum Aplikasi}
	Aplikasi pengumpulan data yang dibuat merupakan pengembangan terhadap aplikasi \textit{spreadsheet} kolaboratif yang sudah ada sebelumnya yakni EtherCalc. Aplikasi yang ditambahkan ini yang akan melakukan pengaturan koneksi ke basis data. Saat pengguna menginginkan penyimpanan data ke dalam basis data yang dituju, aplikasi ini akan melakukan pencarian bagian label dan data dan melakukan validasi masukan sebelum memasukkan data ke dalam basis data. 

	\subsection{Spesifikasi Kebutuhan}
	Pada subbab ini akan dipaparkan \textit{use case} aplikasi yang akan dibuat serta kebutuhan fungsional dan non-fungsional dari aplikasi. Kasus penggunaan oleh pengguna diberi ID dengan format UC-XX dengan UC menyatakan \textit{use case} dan XX menyatakan nomor. Pengguna adalah pihak yang menggunakan aplikasi \textit{spreadsheet} yang sudah ditambahkan fitur pengumpulan data. Kasus penggunaan oleh pengguna dijelaskan pada Tabel \ref{KebutuhanPengguna}.

	\begin{longtable}{ | p{2cm} | p{10cm} | }
	    \caption{Kasus Penggunaan oleh Pengguna}
	    \label{KebutuhanPengguna}\\ \hline
	    \centering\bfseries{ID} & \centering\bfseries{Keterangan} \tabularnewline \hline
	    \endfirsthead
	    \hline
	    \centering\bfseries{ID} & \centering\bfseries{Keterangan} \tabularnewline \hline
	    \endhead
	    UC-01 & Pengguna dapat menentukan basis data tujuan dengan konfigurasi basis data yang diinginkan. \\ \hline
	    UC-02 & Pengguna dapat memuat \textit{spreadsheet} serta menyimpan data ke dalam basis data saat dibutuhkan. \\ \hline
	    UC-03 & Pengguna dapat memperbaiki atribut yang terdeteksi secara otomatis. \\ \hline
	    UC-04 & Pengguna dapat memberikan batasan dan validasi pada suatu domain data. \\ \hline
	\end{longtable}

	Berdasarkan kasus penggunaan di atas, dirancang kebutuhan fungsional perangkat lunak yang diberi ID dengan format FR-XX dengan FR merupakan singkatan dari \textit{functional requirement} dan XX menyatakan nomor kebutuhan. Kebutuhan fungsional dijelaskan pada Tabel \ref{KebutuhanFungsional}.

	\begin{longtable}{ | p{2cm} | p{6cm} | p{4cm} | }
	    \caption{Kebutuhan Fungsional Aplikasi}
	    \label{KebutuhanFungsional}\\ \hline
	    \centering\bfseries{ID} & \centering\bfseries{Keterangan} & \centering\bfseries{ID Use Case Terkait} \tabularnewline \hline
	    \endfirsthead
	    \hline
	    \centering\bfseries{ID} & \centering\bfseries{Keterangan} & \centering\bfseries{ID Use Case Terkait} \tabularnewline \hline
	    \endhead
	    FR-01 & Aplikasi dapat melakukan koneksi kepada basis data yang ditentukan oleh pengguna melalui data masukan berupa \textit{host}, \textit{port}, \textit{username}, \textit{password}, dan \textit{database} dari basis data yang dituju. & UC-01 \\ \hline
	    FR-02 & Aplikasi dapat melakukan perintah basis data kepada basis data yang dituju. & UC-01, UC-02 \\ \hline
	    FR-03 & Aplikasi menyediakan tombol untuk melakukan \textit{commit} terhadap data yang akan disimpan. & UC-02 \\ \hline
	    FR-04 & Aplikasi dapat menampilkan hasil identifikasi label dan data & UC-03 \\ \hline
	    FR-05 & Aplikasi menyediakan fitur bagi pengguna agar dapat mengubah hasil identifikasi label dan data & UC-03 \\ \hline
	    FR-06 & Aplikasi menyediakan fitur bagi pengguna agar dapat menambahkan batasan masukan pada suatu data & UC-04 \\ \hline
	    FR-07 & Aplikasi dapat melakukan validasi data masukan sesuai dengan batasan yang diberikan oleh pengguna & UC-04 \\ \hline
	\end{longtable}

	Selain kebutuhan fungsional, dijabarkan juga kebutuhan non-fungsional yang memiliki ID dengan format NF-XX dengan NF merupakan singkatan dari \textit{non-functional requrirement} dan XX menyatakan nomor. Kebutuhan non-fungsional disajikan pada Tabel \ref{KebutuhanNonfungsional}.

	\begin{longtable}{ | p{2cm} | p{6cm} | p{4cm} | }
	    \caption{Kebutuhan Non-fungsional Aplikasi}
	    \label{KebutuhanNonfungsional}\\ \hline
	    \centering\bfseries{ID} & \centering\bfseries{Keterangan} & \centering\bfseries{ID Use Case Terkait} \tabularnewline \hline
	    \endfirsthead
	    \hline
	    \centering\bfseries{ID} & \centering\bfseries{Keterangan} & \centering\bfseries{ID Use Case Terkait} \tabularnewline \hline
	    \endhead
	    NF-01 & Data masukan pengguna disimpan secara persisten. & UC-01, UC-02 \\ \hline
	    NF-02 & Aplikasi dapat berjalan diatas aplikasi \textit{spreadsheet} EtherCalc & - \\ \hline
	\end{longtable}

	\subsection{Kebutuhan Modul} \label{KebutuhanModul}
	Pembangunan fitur ini diatas aplikasi EtherCalc terdiri dari lima buah modul, yaitu:
	\begin{enumerate}
		\item Modul \texttt{player}, bertugas sebagai jembatan antara \textit{front-end} dan \textit{back-end} dari fitur.
		\item Modul \texttt{db}, bertugas untuk antarmuka baca tulis basis data.
		\item Modul \texttt{framefinder}, bertugas untuk mendeteksi secara otomatis bagian label dan data pada tabel.
		\item Modul \texttt{hierarchyfinder}, bertugas untuk mendeteksi secara otomatis tabel-tabel yang ada dalam suatu \textit{sheet}.
		\item Modul \texttt{checker}, bertugas untuk melakukan pengecekan data sebelum diteruskan ke basis data.
	\end{enumerate}

	Ketergantungan antar modul dapat dilihat pada Gambar \ref{ModuleDependency}

	\begin{figure}[htb]
	    \centering
	    \includegraphics[width=0.6\textwidth]{resources/chapter-4-module-dependecy.png}
	    \caption{Ketergantungan Antar Modul}
		\label{ModuleDependency}
	\end{figure}

	\subsection{Kolaborasi Antar Modul}
	Proses fitur ini akan dilakukan melalui modul \texttt{player} yang dapat menerima perintah pengguna melalui \textit{front-end}. Selanjutnya modul \texttt{framefinder} akan melakukan pendeteksian label dan data secara otomatis pada masing-masing tabel yang terdapat pada \textit{sheet}. Tabel-tabel tersebut didapatkan melalui modul \texttt{hierarchyfinder}. Selanjutnya, pada saat menerima perintah penyimpanan, modul \texttt{checker} akan dipanggil oleh \texttt{player}. Jika data masukan sudah benar, maka modul \texttt{db} akan melakukan pennyimpanan ke dalam basis data. Kolaborasi antar modul disajikan pada Gambar \ref{ModuleFlow}.

	\begin{figure}[htb]
	    \centering
	    \includegraphics[width=0.4\textwidth]{resources/chapter-4-module-flow.png}
	    \caption{Kolaborasi Antar Modul}
		\label{ModuleFlow}
	\end{figure}


\section{Implementasi}
Implementasi dilakukan dengan membangun modul yang telah dijabarkan pada Subbab \ref{KebutuhanModul} dengan menggunakan bahasa Javascript, menyesuaikan dengan modul lain yang telah ada pada aplikasi EtherCalc.
	\subsection{Modul Player}
	Modul \texttt{player} merupakan modul yang menjembatani masukan pengguna dari yang berasal dari \textit{front-end} sehingga dapat diterima oleh modul yang berada di \textit{back-end}. Modul ini hanya terdiri dari satu kelas utama yakni kelas \texttt{player} yang berisi fungsi-fungsi yang dapat dipanggil oleh \textit{front-end} yang dapat dilihat pada Tabel \ref{FungsiModulPlayer}.

	\begin{longtable}{ | p{2cm} | p{10cm} | }
	    \caption{Fungsi pada Kelas \texttt{Player}}
	    \label{FungsiModulPlayer}\\ \hline
	    \centering\bfseries{Fungsi} & \centering\bfseries{Keterangan} \tabularnewline \hline
	    \endfirsthead
	    \hline
	    \centering\bfseries{Fungsi} & \centering\bfseries{Keterangan} \tabularnewline \hline
	    \endhead
	    refresh & Melakukan pembaharuan tampilan.\\ \hline
	    save & Melakukan pemanggilan terhadap modul \texttt{checker} dan melakukan penyimpanan ke basis data. \\ \hline
	    scan & Melakukan identifikasi tabel melalui pemanggilan modul \texttt{framefinder} yang selanjutnya akan menampilkan hasil identifikasi dan kolom perubahan konfigurasi yang dapat diisi pengguna. \\ \hline
	    saveConfig & Melakukan penyimpanan konfigurasi tabel yang dilakukan oleh pengguna. \\ \hline
	    connect & Melakukan koneksi ke basis data yang dipilih. \\ \hline
	\end{longtable}
	(( jelasin yang menerima input pengguna pake script ))

	(( jelasin alur pemanggilan kelas-kelas / API ))

	\subsection{Modul DB}
	Modul basis data digunakan sebagai antarmuka modul lain untuk melakukan operasi I/O basis data. Pada Tugas Akhir ini, basis data yang digunakan adalah MySQL. Modul ini hanya terdiri dari satu kelas utama yakni kelas \texttt{db}. Kelas ini memiliki tugas sebagai penghubung aplikasi ke basis data MySQL yang dipilih. Fungsi-fungsi yang terdapat pada kelas ini dapat dilihat pada Tabel \ref{FungsiModulDB}.

	\begin{longtable}{ | p{2cm} | p{10cm} | }
	    \caption{Fungsi pada Kelas \texttt{DB}}
	    \label{FungsiModulDB}\\ \hline
	    \centering\bfseries{Fungsi} & \centering\bfseries{Keterangan} \tabularnewline \hline
	    \endfirsthead
	    \hline
	    \centering\bfseries{Fungsi} & \centering\bfseries{Keterangan} \tabularnewline \hline
	    \endhead
	    createTable & Fungsi yang digunakan untuk membuat Table tempat pengisian data.\\ \hline
	    isTableExists & Melakukan pengecekan ada atau tidaknya tabel tersebut pada basis data.\\ \hline
	    dropTable & Menghapus tabel yang dipilih.\\ \hline
	    insertData & Memasukkan data ke dalam tabel yang dipilih.\\ \hline
	\end{longtable}

	Tabel akan dibuat pada basis data yang ditentukan, setiap tabel merepresentasikan suatu tabel pada \textit{spreadsheet} yang ditentukan oleh pengguna. \textit{Header} yang didefinisikan oleh pengguna pada \textit{spreadsheet} akan dijadikan \textit{column} pada tabel basis data, tipe yang dibentuk mengikuti masukan pengguna. Tiap baris data yang ada dibawah \textit{header} pada \textit{spreadsheet} akan ditranslasikan menjadi bentuk relasional agar dapat dimasukan ke dalam tabel.
	
	\subsection{Modul Hierarchyfinder}
	Modul \texttt{hierarchyfinder} menggunakan algoritma \textit{hierarchical clustering} untuk dapat mengetahui mana yang merupakan suatu kesatuan tabel pada suatu \textit{sheet}. Modul ini dapat menentukan tabel-tabel yang terdapat pada suatu \textit{sheet} yang selanjutnya akan dilakukan identifikasi label oleh modul \texttt{framefinder}. Algoritma \textit{hierarchical clustering} yang digunakan menganggap setiap sel pada \textit{spreadsheet} merupakan suatu node. Sel-sel yang bersebelahan dengan sel tersebut akan dianggap tetangga sehingga memiliki jarak sama dengan 0. Sel yang digabungkan dengan sel lain akan dihitung sebagai satu \textit{node}. Aturan perhitungan jarak antar \textit{node} dapat dilihat pada kode di Kode \ref{KodeJarak}.\\

	\begin{lstlisting}[frame=single, basicstyle=\linespread{1}\scriptsize\listingsfont, captionpos=b, caption={Perhitungan Jarak \textit{Node}}, label=KodeJarak]
	# Fungsi cellDistance(v1, v2)
	# Parameter pada fungsi adalah v1 (node 1) dan v2 (node 2)
	t = new Table null, null
	colD = t.GetCellCol(v1[0]) - t.GetCellCol(v2[0])
	rowD = t.GetCellRow(v1[0]) - t.GetCellRow(v2[0])

	# Jika sel saling bertetangga tetapi bukan secara diagonal
	# Jika bertetangga, jarak kedua sel adalah 0
	if colD == 0
		if rowD == 1 or rowD == -1
			return 0
	if rowD == 0
		if colD == 1 or colD == -1
			return 0

	# Jika tidak, cek apakah sel bertetangga secara diagonal
	# Jika bertetangga, jarak kedua sel adalah 0
	leftTop = [[v1[1], v1[2]], [v2[1], v2[2]]]
	leftBot = [[v1[1], (v1[2] + v1[4])], [v2[1], (v2[2] + v2[4])]]
	righTop = [[(v1[1] + v1[3]), v1[2]], [(v2[1] + v2[3]), v2[2]]]
	righBot = [[(v1[1] + v1[3]), (v1[2] + v1[4])], [(v2[1] + v2[3]), (v2[2] + v2[4])]]

	if (leftTop[0][0] == righTop[1][0] and leftTop[0][1] == righTop[1][1])
		return 0
	if (leftBot[0][0] == righBot[1][0] and leftBot[0][1] == righBot[1][1])
		return 0

	if (leftTop[1][0] == righTop[0][0] and leftTop[1][1] == righTop[0][1])
		return 0
	if (leftBot[1][0] == righBot[0][0] and leftBot[1][1] == righBot[0][1])
		return 0

	if (leftTop[0][0] == leftBot[1][0] and leftTop[0][1] == leftBot[1][1])
		return 0
	if (righTop[0][0] == righBot[1][0] and righTop[0][1] == righBot[1][1])
		return 0

	if (leftTop[1][0] == leftBot[0][0] and leftTop[1][1] == leftBot[0][1])
		return 0
	if (righTop[1][0] == righBot[0][0] and righTop[1][1] == righBot[0][1])
		return 0

	# Jika tidak juga, hitung jarak menggunakan teknik Euclidian
	dist = Math.sqrt(Math.pow((v2[1] + (v2[3]/2)) - (v1[1] + (v1[3]/2)), 2) + Math.pow((v2[2] + (v2[4]/2)) - (v1[2] + (v1[4]/2)), 2));
	return dist
	\end{lstlisting}

	Hasil dari pengelompokan ini berupa kelompok-kelompok tabel pada \textit{spreadsheet}. Setiap tabel yang teridentifikasi selanjutnya akan dicari bagian \textit{header} dan \textit{label} menggunakan modul \texttt{framefinder}.

	\subsection{Modul Framefinder}
	Modul \texttt{framefinder} melakukan pengidentifikasian terhadap tabel yang ada sehingga dapat diketahui baris yang merupakan \textit{header} dan \textit{data}. Implementasi modul ini dilakukan dengan mengikuti implementasi yang dilakukan pada penelitian yang dilakukan oleh Chen \citep{Chen2013}. Modul ini terdiri dari 5 kelas yang dapat dilihat pada Tabel \ref{KelasModulFF}.

	\begin{longtable}{ | p{3cm} | p{10cm} | }
	    \caption{Kelas pada Modul \texttt{FrameFinder}}
	    \label{KelasModulFF}\\ \hline
	    \centering\bfseries{Nama Kelas} & \centering\bfseries{Keterangan} \tabularnewline \hline
	    \endfirsthead
	    \hline
	    \centering\bfseries{Nama Kelas} & \centering\bfseries{Keterangan} \tabularnewline \hline
	    \endhead
	    LoadSheet & Kelas ini berfungsi sebagai kelas yang melakukan pengambilan data dan konversi sel dan \textit{sheet} pada \textit{spreadsheet} ke dalam bentuk kelas-kelas yang ada pada modul ini.\\ \hline
	    MySheet & Merupakan kelas bentukan yang merepresentasikan \textit{sheet} pada \textit{spreadsheet} yang dipilih.\\ \hline
	    MyCell & Merupakan kelas untuk merepresentasikan \textit{properties} yang ada pada sel pada \textit{sheet} yang dipilih.\\ \hline
	    FeatureSheetRow & Melakukan ekstraksi fitur-fitur yang terdapat pada suatu \textit{sheet} pada \textit{spreadsheet}.\\ \hline
	    PredictSheetRows & Kelas ini digunakan untuk menghasilkan file dalam format yang dapat dibaca oleh algoritma Conditional Random Field (CRF) dari fitur-fitur yang telah diekstraksi pada \textit{sheet}.\\ \hline
	\end{longtable}

	\subsubsection{Kelas LoadSheet}
	Kelas ini melakukan pengambilan data pada \textit{spreadsheet} dengan cara membaca file \textit{spreadsheet} dan membentuk representasi kelas yang dibutuhkan. Kelas ini memiliki satu atribut utama yakni cmysheet yang merupakan kelas MySheet. Fungsi-fungsi yang terdapat pada kelas ini dapat dilihat pada Tabel \ref{FungsiLoadSheet}.

	\begin{longtable}{ | p{4cm} | p{9cm} | }
	    \caption{Fungsi pada Kelas LoadSheet}
	    \label{FungsiLoadSheet}\\ \hline
	    \centering\bfseries{Nama Fungsi} & \centering\bfseries{Keterangan} \tabularnewline \hline
	    \endfirsthead
	    \hline
	    \centering\bfseries{Nama Fungsi} & \centering\bfseries{Keterangan} \tabularnewline \hline
	    \endhead
	    loadSheetDict & Fungsi ini merupakan fungsi utama yang bertugas untuk membuat representasi \textit{spreadsheet} yang diterima ke dalam kelas MySheet.\\ \hline
	    getValueType & Untuk mendapatkan tipe representasi data yang diberikan oleh \textit{spreadsheet}. Contoh: tanggal, nominal uang, desimal, dan lain-lain.\\ \hline
	    getDataType & Untuk mendapatkan tipe data primitif pada suatu sel.\\ \hline
	    featureIndentation & Digunakan untuk mengecek keberadaan \textit{property} \textit{indentation} pada sel.\\ \hline
	    featureAlignStyle & Digunakan untuk mengecek keberadaan \textit{property} \textit{align} pada sel\\ \hline
	    featureFontBold & Digunakan untuk mengecek keberadaan \textit{property} \textit{bold} pada sel\\ \hline
	    featureFontHeight & Digunakan untuk mengecek keberadaan \textit{property} \textit{height} pada sel\\ \hline
	    featureFontUnderline & Digunakan untuk mengecek keberadaan \textit{property} \textit{underline} pada sel\\ \hline
	    featureFontItalic & Digunakan untuk mengecek keberadaan \textit{property} \textit{italic} pada sel\\ \hline
	    featureFontBgcolor & Digunakan untuk mengecek keberadaan \textit{property} \textit{background color} pada sel\\ \hline
	    fFeatureBorderStyle & Digunakan untuk mengecek keberadaan \textit{property} \textit{border} pada sel.\\ \hline
	\end{longtable}

	\subsubsection{Kelas MySheet}
	Kelas MySheet merupakan kelas yang merepresentasikan \textit{sheet} pada \textit{spreadsheet} yang dipilih. Kelas ini memiliki 4 atribut yang dapat dilihat pada Tabel \ref{AtributMySheet}.

	\begin{longtable}{ | p{3cm} | p{8cm} | }
	    \caption{Atribut pada Kelas MySheet}
	    \label{AtributMySheet}\\ \hline
	    \centering\bfseries{Nama Atribut} & \centering\bfseries{Keterangan} \tabularnewline \hline
	    \endfirsthead
	    \hline
	    \centering\bfseries{Nama Atribut} & \centering\bfseries{Keterangan} \tabularnewline \hline
	    \endhead
	    sheetdict & Merupakan representasi kumpulan sel-sel pada suatu \textit{sheet}. Tiap sel direpresentasikan dalam bentuk kelas MyCell.\\ \hline
	    mergerowdict & Merupakan kumpulan sel-sel yang digabungkan.\\ \hline
	    maxcolnum & Nilai kolom terbesar pada sel.\\ \hline
	    maxrownum & Nilai baris terbesar pada sel.\\ \hline
	\end{longtable}

	Pada kelas ini terdapat 3 fungsi yang dapat dilihat pada Tabel \ref{FungsiMySheet}.

	\begin{longtable}{ | p{4cm} | p{9cm} | }
	    \caption{Fungsi pada Kelas MySheet}
	    \label{FungsiMySheet}\\ \hline
	    \centering\bfseries{Nama Fungsi} & \centering\bfseries{Keterangan} \tabularnewline \hline
	    \endfirsthead
	    \hline
	    \centering\bfseries{Nama Fungsi} & \centering\bfseries{Keterangan} \tabularnewline \hline
	    \endhead
	    getCellsArray & Digunakan untuk mendapatkan seluruh refresentasi sel pada kelas ini dalam bentuk \textit{array}.\\ \hline
	    addMergeCell & Digunakan pada saat terdapat sel yang digabungkan. Sel tersebut akan dimasukkan ke dalam daftar \textit{merged cells}.\\ \hline
	    insertCell & Menambahkan sel ke dalam kelas ini. Sel yang ditambahkan akan direpresentasikan dalam bentuk kelas MyCell.\\ \hline
	\end{longtable}

	\subsubsection{Kelas MyCell}
	Kelas MyCell merupakan kelas yang merepresentasikan sel pada suatu \textit{sheet}. Kelas ini memiliki 17 atribut yang dapat dilihat pada Tabel \ref{AtributMyCell}.

	\begin{longtable}{ | p{3cm} | p{8cm} | }
	    \caption{Atribut pada Kelas MyCell}
	    \label{AtributMyCell}\\ \hline
	    \centering\bfseries{Nama Atribut} & \centering\bfseries{Keterangan} \tabularnewline \hline
	    \endfirsthead
	    \hline
	    \centering\bfseries{Nama Atribut} & \centering\bfseries{Keterangan} \tabularnewline \hline
	    \endhead
	    x & Merupakan letak sel pada koordinat X.\\ \hline
	    y & Merupakan letak sel pada koordinat Y. \\ \hline
	    w & Nilai lebar sel.\\ \hline
	    h & Nilai tinggi sel.\\ \hline
	    cstr & Isi sel dalam bentuk \textit{string}.\\ \hline
	    mtype & Tipe konten yang ada di dalam sel.\\ \hline
	    indents & Nilai indentasi jika terdapat indentasi pada konten.\\ \hline
	    centeralign & Bernilai \textit{true} atau \textit{false} bergantung pada \textit{align} sel merupakan rata tengah atau tidak.\\ \hline
	    leftalign & Bernilai \textit{true} atau \textit{false} bergantung pada \textit{align} sel merupakan rata kiri atau tidak.\\ \hline
	    rightalign & Bernilai \textit{true} atau \textit{false} bergantung pada \textit{align} sel merupakan rata kanan atau tidak.\\ \hline
	    bottomborder & Bernilai \textit{true} atau \textit{false} bergantung pada \textit{property} \textit{bottom border} ada atau tidak.\\ \hline
	    upperborder & Bernilai \textit{true} atau \textit{false} bergantung pada \textit{property} \textit{upper border} ada atau tidak.\\ \hline
	    leftborder & Bernilai \textit{true} atau \textit{false} bergantung pada \textit{property} \textit{left border} ada atau tidak.\\ \hline
	    rightborder & Bernilai \textit{true} atau \textit{false} bergantung pada \textit{property} \textit{right border} ada atau tidak.\\ \hline
	    bold & Bernilai \textit{true} atau \textit{false} bergantung pada \textit{property} \textit{bold} ada atau tidak.\\ \hline
	    italic & Bernilai \textit{true} atau \textit{false} bergantung pada \textit{property} \textit{italic} ada atau tidak.\\ \hline
	    underline & Bernilai \textit{true} atau \textit{false} bergantung pada \textit{property} \textit{underline} ada atau tidak.\\ \hline
	\end{longtable}

	% Pada kelas ini terdapat 3 fungsi yang dapat dilihat pada Tabel \ref{FungsiMyCell}.

	% \begin{longtable}{ | p{4cm} | p{9cm} | }
	%     \caption{Fungsi pada Kelas MyCell}
	%     \label{FungsiMyCell}\\ \hline
	%     \centering\bfseries{Nama Fungsi} & \centering\bfseries{Keterangan} \tabularnewline \hline
	%     \endfirsthead
	%     \hline
	%     \centering\bfseries{Nama Fungsi} & \centering\bfseries{Keterangan} \tabularnewline \hline
	%     \endhead
	%     writestrAlignstyle & Mendapatkan representasi \textit{string} untuk \textit{align} pada sel. Digunakan pada saat penulisan ke dalam file untuk dibaca pada algoritma CRF.\\ \hline
	%     writestrBorderstyle & Mendapatkan representasi \textit{string} untuk \textit{border} pada sel. Digunakan pada saat penulisan ke dalam file untuk dibaca pada algoritma CRF.\\ \hline
	%     getIndents & Digunakan untuk menentukan nilai atribut indentasi dengan menghitung seberapa dalam indentasi pada konten.\\ \hline
	% \end{longtable}

	\subsubsection{Kelas FeatureSheetRow}
	Kelas ini bertugas untuk melakukan ekstraksi fitur pada tiap baris sel yang telah dikumpulkan dari \textit{spreadsheet}. Fitur-fitur yang diambil untuk setiap barisnya dapat dilihat pada Tabel \ref{FiturEkstraksi}.

	\begin{longtable}{ | p{10cm} | }
	    \caption{Fitur yang Diambil dari Sel}
	    \label{FiturEkstraksi}\\ \hline
	    \centering\bfseries{Fitur} \tabularnewline \hline
	    \endfirsthead
	    \hline
	    \centering\bfseries{Fitur} \tabularnewline \hline
	    \endhead
	    Baris memiliki sel yang digabung \\ \hline
	    Sel pada baris mencapai kolom paling kanan \\ \hline
	    Sel pada baris mencapai kolom paling kiri \\ \hline
	    Baris hanya memiliki 1 kolom \\ \hline
	    Baris memiliki sel rata tengah \\ \hline
	    Baris memiliki sel rata kiri \\ \hline
	    Baris memiliki sel yang ditebalkan (\textit{bold}) \\ \hline
	    Baris memiliki sel berindentasi \\ \hline
	    Baris memiliki sel berisi kata `Table` \\ \hline
	    Baris memiliki sel berisi kata berawalan tanda baca \\ \hline
	    Baris memiliki sel dengan presentase angka tinggi \\ \hline
	    Baris memiliki sel berisi huruf besar seluruhnya \\ \hline
	    Baris memiliki sel berisi kata dengan awal huruf besar \\ \hline
	    Baris memiliki sel berisi kata dengan awal huruf kecil \\ \hline
	    Baris memiliki persentase sel memiliki isi tinggi \\ \hline
	    Baris memiliki persentase sel memiliki isi berupa kata tinggi \\ \hline
	    Baris memiliki sel berisi karakter spesial \\ \hline
	    Baris memiliki sel berisi karakter titik koma \\ \hline
	    Baris memiliki jumlah sel berisi tahun tinggi \\ \hline
	    Baris memiliki persentase sel berisi tahun tinggi \\ \hline
	    Baris memiliki jumlah sel berisi kata dengan huruf yang banyak tinggi \\ \hline
	\end{longtable}

	Fitur-fitur diatas mengikuti fitur yang didefinisikan pada penelitian yang dilakukan oleh Chen \citep{Chen2013}. Fitur-fitur ini akan digunakan dalam perhitungan algoritma CRF.

	\subsubsection{Kelas PredictSheetRows}
	Kelas PredictSheetRows memiliki tugas untuk melakukan konversi fitur-fitur yang telah didapatkan pada kelas FeatureSheetRow menjadi bentuk file teks yang dapat dibaca oleh program CRFPP yang akan menjalankan algoritma CRF pada file tersebut. Contoh file yang dihasilkan oleh kelas ini dapat dilihat pada Kode \ref{KodeFile}.\\

	\begin{lstlisting}[frame=single, basicstyle=\linespread{1}\scriptsize\listingsfont, captionpos=b, caption={File Berisikan Fitur Tiap Baris}, label=KodeFile]
	DeadlineTA.xls____Sheet1____1 1 0 1 0 0 1 0 1 0 0 0 0 0 0 1 0 1 1 0 0 0 0 0 Header
	DeadlineTA.xls____Sheet1____2 0 1 1 0 0 1 0 0 0 0 0 1 0 0 1 0 1 1 0 0 0 0 0 Data
	DeadlineTA.xls____Sheet1____3 0 1 1 0 0 1 0 0 0 0 0 1 0 0 1 0 0 0 0 0 0 0 0 Data
	DeadlineTA.xls____Sheet1____4 0 1 1 0 0 1 0 0 0 0 0 1 0 0 1 0 0 0 0 0 0 0 0 Data
	DeadlineTA.xls____Sheet1____5 0 1 1 0 0 1 0 0 0 0 0 1 0 0 1 0 0 1 0 0 0 0 0 Data
	\end{lstlisting}

	File tersebut ditulis dalam format \texttt{namafile\_namasheet\_bariske [fitur-fitur pada baris] label}. File ini yang akan diolah oleh algoritma dan digunakan untuk memprediksi label dari setiap baris tersebut. Algoritma yang digunakan adalah Conditional Random Field dan menggunakan aplikasi CRFPP yang berjalan eksternal diluar kelas ini untuk membaca file, membaca model, serta memprediski label.

	\subsection{Modul Checker}
	Modul \texttt{checker} memiliki tugas untuk melakukan pengecekan terhadap masukan pengguna pada konfigurasi serta melakukan pengecekan kesesuaian nilai masukan berdasarkan tipe, \textit{range}, serta relasi yang ditentukan oleh pengguna. Jika seluruh kriteria telah dipenuhi, maka modul ini akan memanggil modul \texttt{db} untuk melakukan penyimpanan data. (Jelasin checkingnya)

	\subsection{Antarmuka}

\section{Pengujian}
% \blindtext

\section{Pembahasan}
% \blindtext