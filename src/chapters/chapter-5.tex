\chapter{Penutup}
Bab ini berisi hal-hal yang dapat disimpulkan dari pelaksanaan Tugas Akhir ini. Bab ini juga mencakup saran untuk pengembangan Tugas Akhir ini di masa mendatang.

\section{Kesimpulan}
Berdasarkan hasil pengembangan aplikasi pengumpulan data menggunakan \textit{spreadsheet} yang telah dilakukan. Berikut adalah kesimpulan yang diperoleh.
\begin{enumerate}
	\item Identifikasi tabel pada suatu \textit{sheet} dapat dilakukan dengan menggunakan algoritma hierarchical clustering.
	\item Identifikasi label suatu baris pada tabel dapat dilakukan dengan teknik framefinder dengan membagi label menjadi empat jenis yakni \textit{title}, \textit{data}, \textit{header}, dan \textit{footer}
	\item Penyimpanan data dari bentuk \textit{spreadsheet} ke dalam bentuk basis data dapat dilakukan dengan cepat dan diharapkan dapat mempercepat proses pengumpulan data.
\end{enumerate}

\section{Saran}
Saran yang dapat diberikan untuk pengembangan di masa mendatang adalah sebagai berikut:
\begin{enumerate}
	\item Pada pembangunan selanjutnya dapat ditambahkan fitur untuk menangani kasus tabel yang lebih rumit seperti formulir.
	\item Penambahan data pembelajaran untuk identifikasi label baris dapat dilakukan sehingga akan memperbaiki hasil identifikasi otomatis. Selain itu pada pengembangan selanjutnya dapat ditambahkan \textit{feedback} dari pengguna sebagai data pembelajaran.
\end{enumerate}