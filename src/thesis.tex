%--------------------------------------------------------------------%
%
% Berkas utama templat LaTeX.
%
% author Petra Barus, Peb Ruswono Aryan
%
%--------------------------------------------------------------------%
%
% Berkas ini berisi struktur utama dokumen LaTeX yang akan dibuat.
%
%--------------------------------------------------------------------%

\documentclass[11pt, a4paper, onecolumn, oneside, final]{book}

%-------------------------------------------------------------------%
%
% Konfigurasi dokumen LaTeX untuk laporan tesis IF ITB
%
% @author Petra Novandi
%
%-------------------------------------------------------------------%
%
% Berkas asli berasal dari Steven Lolong
%
%-------------------------------------------------------------------%

% Ukuran kertas
\special{papersize=210mm,297mm}

% Setting margin
\usepackage[top=3cm,bottom=3cm,left=4cm,right=3cm]{geometry}

\usepackage{mathptmx}

% Judul bahasa Indonesia
\usepackage[bahasa]{babel}

% Format citation
\usepackage[backend=bibtex,citestyle=authoryear]{biblatex}

\usepackage[utf8]{inputenc}
\usepackage{csquotes}
\usepackage{graphicx}
\usepackage{titling}
\usepackage{blindtext}
\usepackage{sectsty}
\usepackage{chngcntr}
\usepackage{etoolbox}
\usepackage{hyperref}       % Package untuk link di daftar isi.
\usepackage{titlesec}       % Package Format judul
\usepackage{parskip}
\usepackage{pgfgantt}		
\usepackage{ragged2e}
\usepackage{libertine}
\usepackage{booktabs}
\usepackage{longtable}
\usepackage{array}
\usepackage{caption}
\usepackage{enumitem}

\captionsetup[longtable]{skip=0.1em}
\setlist{parsep=0pt,listparindent=\parindent}

% Line satu setengah spasi
\renewcommand{\baselinestretch}{1.5}

% Add comma between Author and Year
\renewcommand*{\nameyeardelim}{\addcomma\space}

% Setting judul
\chapterfont{\centering \Large}
\titleformat{\chapter}[display]
  {\Large\centering\bfseries}
  {\chaptertitlename\ \thechapter}{0pt}
    {\Large\bfseries\uppercase}
\titlespacing*{\chapter}{0pt}{-50pt}{40pt}

% Setting nomor pada subbsubsubbab
\setcounter{secnumdepth}{4}

\titleformat{\paragraph}
{\normalfont\normalsize\bfseries}{\theparagraph}{1em}{}
\titlespacing*{\paragraph}
{0pt}{3.25ex plus 1ex minus .2ex}{1.5ex plus .2ex}

\makeatletter

\makeatother

% Counter untuk figure dan table.
\counterwithin{figure}{section}
\counterwithin{table}{section}

% For Gantt Chart
\newcounter{myWeekNum}
\stepcounter{myWeekNum}
%
\newcommand{\myWeek}{\themyWeekNum
    \stepcounter{myWeekNum}
    \ifnum\themyWeekNum=53
         \setcounter{myWeekNum}{1}
    \else\fi
}
%
\newcolumntype{L}[1]{>{\raggedright\let\newline\\\arraybackslash\hspace{0pt}}m{#1}}
\newcolumntype{C}[1]{>{\centering\let\newline\\\arraybackslash\hspace{0pt}}m{#1}}
\newcolumntype{R}[1]{>{\raggedleft\let\newline\\\arraybackslash\hspace{0pt}}m{#1}}

\makeatletter

\makeatother

%\bibliography{references}

\begin{document}

    %Basic configuration
    \title{Pengembangan Aplikasi Pengumpulan Data Menggunakan \textit{Spreadsheet}}
    \date{}
    \author{
        Feryandi Nurdiantoro \\
        NIM 13513042
    }   

    \frontmatter
    \clearpage
\pagestyle{empty}

\begin{center}
\smallskip

    \Large \bfseries \MakeUppercase{\thetitle}
    \vfill

    \Large Laporan Tugas Akhir 1
    \vfill

    \large Disusun sebagai syarat kelulusan IF4091
    \vfill

    \large Oleh

    \Large \theauthor

    \vfill
    \begin{figure}[h]
        \centering
            \includegraphics[width=0.15\textwidth]{resources/cover-ganesha.jpg}
    \end{figure}
    \vfill

    \large
    \uppercase{
        Program Studi Teknik Informatika \\
        Sekolah Teknik Elektro dan Informatika \\
        Institut Teknologi Bandung
    }

    Desember 2016

\end{center}

\clearpage

    \clearpage
\pagestyle{empty}

\begin{center}
\smallskip

    \Large \bfseries \MakeUppercase{\thetitle}
    \vfill

    \Large Laporan Tugas Akhir 1
    \vfill

    \large Oleh

    \Large \theauthor

    \large Program Studi Teknik Informatika \\
    Sekolah Teknik Elektro dan Informatika \\
    Institut Teknologi Bandung \\

    \vfill
    \normalsize \normalfont
    Telah disetujui dan disahkan sebagai Laporan Tugas Akhir 1 di Bandung, pada tanggal     Desember 2016.

    \vfill
    \setlength{\tabcolsep}{12pt}
    \begin{tabular}{c@{\hskip 0.5in}c}
        Pembimbing I, & Pembimbing II, \\
        & \\
        & \\
        & \\
        & \\
        Tricya Esterina Widagdo, ST., M.Sc. & Yudistira Dwi Wardhana Asnar, Ph.D \\
        NIP 197109071997022001 & NIP 999126502 \\
    \end{tabular}

\end{center}
\clearpage

    %\input{chapters/statement}

    \pagestyle{plain}

    \clearpage
\chapter*{ABSTRAK}
\addcontentsline{toc}{chapter}{ABSTRAK}
\begin{center}
\MakeTextUppercase{\textbf{\large{\thetitle}}}

Oleh

\MakeTextUppercase{\theauthor}
\end{center}
\medskip
\begin{spacing}{1.0}
Penggunaan \textit{spreadsheet} di dalam kehidupan sehari-hari tidak terlepas dari pengumpulan data. Hal ini disebabkan oleh mudahnya penggunaan \textit{spreadsheet} sehingga banyak orang awam yang memilih menggunakan \textit{spreadsheet} dibandingkan basis data. Pengumpulan data menggunakan aplikasi \textit{spreadsheet} memiliki beberapa kelemahan seperti tidak adanya validasi, terisolasinya data yang dikumpulkan, serta terdapat kemungkinan sulitnya berkolaborasi didalam pengumpulan data. Sehingga masalah yang ingin diselesaikan disini adalah transformasi data menjadi bentuk basis data, melakukan verifikasi terhadap data, dan dapat dilakukan secara kolaboratif.

Pada laporan ini akan dibahas mengenai cara mentransformasikan data yang terdapat pada \textit{spreadsheet} sehingga dapat diolah ke dalam bentuk basis data. Transformasi ini dilakukan dapat dengan cara otomatis menggunakan algoritma yang pernah diteliti sebelumnya maupun secara manual dengan menggunakan tabel konfigurasi yang disediakan. Validasi akan dilakukan terhadap 3 tipe validasi yakni, tipe data, domain data, dan relasi antar data. Algoritma tersebut akan dibangun diatas \textit{spreadsheet} kolaboratif bernama Ethercalc sebagai solusi untuk dapat melakukan pengumpulan data secara kolaboratif.

Diakhir laporan akan dibahas mengenai hasil percobaan menggunakan perangkat lunak yang telah dibangun. Pengujian dilakukan dengan menggunakan beberapa kasus yang dibuat dan diujikan kebenaran hasil data masukan menjadi data pada basis data.

Kata kunci: \textit{spreadsheet}, pengumpulan data, \textit{data quality}, \textit{data management}.
\end{spacing}

\clearpage
    \input{chapters/abstract-en}
    \chapter*{Kata Pengantar}
\addcontentsline{toc}{chapter}{KATA PENGANTAR}

TBD

Terimakasih kepada Fiqie, Visat

    \titleformat*{\section}{\centering\bfseries\Large\MakeUpperCase}
    
    \tableofcontents
    \addcontentsline{toc}{chapter}{DAFTAR ISI} 
    {%
        \let\oldnumberline\numberline%
        \renewcommand{\numberline}{\figurename~\oldnumberline}%
        \listoffigures%
    }
    \addcontentsline{toc}{chapter}{DAFTAR GAMBAR}  
    {%
        \let\oldnumberline\numberline%
        \renewcommand{\numberline}{\tablename~\oldnumberline}%
        \listoftables%
    }
    \addcontentsline{toc}{chapter}{DAFTAR TABEL}

    %----------------------------------------------------------------%
    % Konfigurasi Bab
    %----------------------------------------------------------------%
    \renewcommand{\chaptername}{BAB}
    \renewcommand{\thechapter}{\Roman{chapter}}
    %----------------------------------------------------------------%

    \titleformat*{\section}{\bfseries\large}
    \mainmatter
    %----------------------------------------------------------------%
    % Dafter Bab
    % Untuk menambahkan daftar bab, buat berkas bab misalnya `chapter-6` di direktori `chapters`, dan masukkan ke sini.
    %----------------------------------------------------------------%
    \chapter{Pendahuluan}

Pada bab ini akan dibahas mengenai gambaran dasar dari pelaksanaan tugas akhir dalam bentuk penjelasan latar belakang yang mendasari pemilihan topik. Dari latar belakang tersebut, akan diurai kembali menjadi rumusan masalah, tujuan, batasan masalah, serta metodologi yang digunakan untuk keperluan Tugas Akhir ini.

\section{Latar Belakang}

Aplikasi \textit{spreadsheet} menjadi aplikasi yang mudah ditemui dan wajar digunakan oleh banyak orang secara personal maupun dalam sebuah organisasi komersial \parencite{Chan1996}. Pada tahun 1979, aplikasi \textit{spreadsheet} pertama dibuat dengan nama VisiCalc. Pengguna komputer pada saat itu dimanjakan dengan kapabilitas dan fleksibilitas aplikasi yang dapat melakukan operasi sederhana tanpa harus menggunakan komputer mainframe. Sejak saat itu, dengan semakin berkembangnya daya komputasi, telah banyak sekali muncul aplikasi \textit{spreadsheet} baru hingga saat ini.

\textit{Spreadsheet} memiliki beberapa keunggulan dibandingkan dengan aplikasi pengolahan data jenis lain. Keunggulan yang paling terlihat adalah banyak orang yang mengetahui cara penggunaan aplikasi jenis \textit{spreadsheet}. Selain itu, \textit{spreadsheet} jika digunakan dengan benar, memiliki banyak fitur dan kemampuan yang jarang diketahui orang awam. Dengan keunggulan ini, \textit{spreadsheet} sering kali dijadikan pilihan utama dalam pengolahan data.

Bagi orang awam, \textit{spreadsheet} sering terlihat sebagai aplikasi yang digunakan untuk melakukan perhitungan sederhana. Selain itu banyak orang menganggap, penggunaan \textit{spreadsheet} adalah personal sehingga tidak membutuhkan tim atau bantuan orang lain dalam pembuatan sebuah \textit{spreadsheet}. Hal ini tidak dapat dibenarkan, karena jika melihat kasus penggunaannya pada organisasi bisnis yang besar, \textit{spreadsheet} yang dihasilkan sangatlah kompleks dan besar dengan pengembangan yang membutuhkan banyak orang \parencite{Panko1998}.

Kompleksitas dan besarnya ukuran \textit{spreadsheet} inilah yang membuat penggunaan \textit{spreadsheet} pada sebuah bisnis sangatlah rentan akan kesalahan. Sebuah kesalahan kecil dapat berakibat fatal dan memberikan kerugian seperti kehilangan pendapatan, kesalahan pemberian harga, penipuan, dan kegagalan sistem akibat ketergantungan berlebih antar \textit{spreadsheet} \parencite{EUSPRIGAbout}. Telah banyak bukti dan penelitian yang menunjukan bahwa kesalahan pada \textit{spreadsheet} sangat mudah ditemui. Bahkan pada \textit{spreadsheet} yang dibuat dengan sangat hati-hati, masih dapat ditemui sekitar 1 persen atau lebih kesalahan pada formula yang dibuat \parencite{Panko1998}.

Tingginya angka kesalahan yang dapat terjadi pada suatu \textit{spreadsheet} merupakan hal yang sangat krusial terutama didalam bisnis. Metode pencegahan harus dapat dilakukan untuk dapat mengurangi angka kesalahan ini. Beberapa organisasi dapat menerapkan pencegahan dengan cara melakukan tahap-tahap metodologi yakni dengan pembuatan desain awal, melakukan metode \textit{best practice} yang tersedia dan sesuai dengan situasi yang dihadapi, menerapkan \textit{policy} khusus pada saat pembuatan \textit{spreadsheet}, melakukan \textit{testing}, serta pembuatan dokumentasi \parencite{EUSPRIGBestPractice}. Namun, metodologi tersebut masih sangat rentan oleh kesalahan manusia karena perangkat lunak yang digunakan tetap tidak diubah didalam menjalankan metodologi tersebut. 

Untuk dapat mengurangi kesalahan-kesalahan yang sering terjadi pada \textit{spreadsheet} secara lebih mendasar, dibutuhkan bantuan perangkat lunak untuk dapat melakukan kontrol terhadap masukan pengguna. Pada tugas akhir ini akan difokuskan pada pengembangan sebuah aplikasi yang dapat membantu pengguna \textit{spreadsheet} mengurangi kesalahan yang sering terjadi pada pengembangan \textit{spreadsheet}.

\section{Rumusan Masalah}

Kesalahan yang cukup banyak terjadi pada penggunaan \textit{spreadsheet} yakni tidak konsistennya \textit{spreadsheet}. Selain itu sering juga terjadi kesalahan memasukan formula atau penggantian fungsi operasi pada \textit{spreadsheet} secara tidak sengaja. Hal-hal tersebut merupakan beberapa masalah utama yang cukup penting untuk diselesaikan terutama didalam penggunaannya pada bisnis dan komersial. Sehingga, berdasarkan latar belakang yang telah disebutkan pada bagian sebelumnya, akan dibentuk sebuah aplikasi yang membantu mengurangi tingkat kesalahan pada \textit{spreadsheet}. Dalam rangka pemabangunan aplikasi, terdapat beberapa permasalahan yang menjadi perhatian pada tugas akhir ini, yaitu:

\begin{enumerate}
    \item Mengetahui jenis kesalahan yang umum dilakukan pada penggunaan \textit{spreadsheet}
    \item Mengetahui kebutuhan dari perangkat lunak berdasarkan jenis-jenis kesalahan yang sering terjadi
    \item Mengetahui rancangan dari perangkat lunak yang dapat memenuhi kebutuhan tersebut
    \item Mengetahui cara mengimplementasi perangkat lunak yang dibutuhkan
\end{enumerate}

\section{Tujuan}

Tujuan yang ingin dicapai dalam Tugas Akhir ini adalah mengembangkan perangkat lunak yang dapat mengurangi tingkat kesalahan pada \textit{spreadsheet} hal ini dilakukan dengan mengintegrasikan basis data pada penggunaan \textit{spreadsheet}. Dengan adanya perangkat lunak ini, diharapkan dapat memvalidasi data masukan dengan lebih baik. Selain itu, dengan adanya perangkat lunak ini diharapkan dapat meningkatkan efisiensi persebaran \textit{spreadsheet} dengan bantuan basis data yang terintegrasi.

\section{Batasan Masalah}

Untuk mencapai tujuan Tugas Akhir ini, terdapat beberapa batasan-batasan tertentu. Batasan tersebut ditujukan untuk memperjelas dan memfokuskan objek penelitian dan pengembangan tugas akhir. Batasan-batasan masalah pengerjaan tugas akhir adalah sebagai berikut.

\begin{enumerate}
    \item Data yang dapat dijadikan masukan berupa data berbentuk tabel atau formulir sederhana.
    \item Struktur basis data yang digunakan dalam penggunaan aplikasi ini dianggap sudah menemenuhi persyaratan normalisasi dan dapat langsung digunakan. Pembentukan basis data yang sesuai tidak dibahas didalam Tugas Akhir ini.
\end{enumerate}

\section{Metodologi}

Metodologi yang digunakan dalam pengerjaan Tugas Akhir ini yakni:
\begin{enumerate}
    \item Studi Literatur

    Pengerjaan tugas akhir diawali dengan mencari dan mempelajari referensi berupa jurnal ilmiah dan aplikasi-aplikasi yang telah ada sebelumnya yang dapat membantu pengembangan aplikasi yang dibuat pada tugas akhir ini. Literatur yang dicari dan dipelajari berkaitan dengan topik tugas akhir yaitu mengenai \textit{spreadsheet}, penggunaannya pada bisnis, kesalahan yang sering dilakukan dalam pembuatan, metode \textit{quality control} yang dapat dilakukan, serta hal-hal lain yang masih berkaitan dengan topik tugas akhir ini. 
    
    \item Analisis Masalah

    Pada tahap ini dilakukan analisis permasalahan yang berkaitan dengan topik yang diangkat pada tugas akhir ini. Selain itu, dilakukan penentuan spesifikasi dan fitur yang ada pada aplikasi tersebut sebagai bentuk solusi terhadap permasalahan yang dianalisis.

    \item Perancangan Solusi

    Pada tahap ini dilakukan perancangan solusi yang dapat menyelesaikan masalah-masalah yang telah dijelaskan pada bagian analisis masalah. Bagian perancangan ini juga menjelaskan arsitektur yang digunakan untuk membangun perangkat lunak berdasarkan spesifikasi dan metode yang digunakan.

    \item Implementasi

    Pada tahap ini dilakukan pembangunan aplikasi sesuai dengan kebutuhan dan spesifikasi dari hasil analisis masalah serta rancangan solusi yang diajukan.

    \item Pengujian dan Analisis Hasil

    Pada tahap ini dilakukan pengujian dengan menggunakan data set uji yang sesuai dengan batasan masalah ke dalam aplikasi yang diimplementasikan. Selanjutnya dilakukan analisis hasil pengujian dan penarikan kesimpulan.

\end{enumerate}

\section{Sistematika Pembahasan}

Penulisan tugas akhir ini terdiri dari 5 bab, yaitu: BAB I Pendahuluan, BAB II Tinjauan Pustaka, BAB III Analisis dan Perancangan, BAB IV Evaluasi dan Pembahasan, dan BAB V Penutup. 

Bab satu membahas mengenai latar belakang permasalahan, rumusan masalah, tujuan, batasan masalah, metodologi serta sistematika pembahasan yang digunakan. Bab ini juga menjelaskan secara umum isi dari tugas besar serta gambaran dasar dari pelaksanaan tugas akhir.

Bab dua menjelaskan mengenai dasar teori yang digunakan didalam menyelesaikan permasalahan yang diangkat. Teori yang digunakan berasal dari literatur dan referensi yang berhubungan dengan permasalahan yang diangkat seperti hal-hal yang berkaitan dengan \textit{spreadsheet}, penggunaannya pada bisnis, kesalahan yang sering dilakukan dalam pembuatan, serta metode \textit{quality control} yang dapat dilakukan untuk mencegah kesalahan. Dasar teori ini menjadi dasar analisis dan rancangan solusi pada bab selanjutnya.

Bab tiga memaparkan analisa kebutuhan dan permasalahan yang dipilih yakni tingginya tingkat kesalahan yang terjadi pada \textit{spreadsheet}. Dari hasil analisa yang dilakukan, di dapatkan bentuk solusi umum yang dapat digunakan untuk mengatasi permasalahan tersebut. Selanjutnya solusi umum tersebut dibuat rancangan dan arsitekturnya agar dapat diimplementasikan.

Bab empat memperlihatkan hasil pengujian yang dilakukan kepada aplikasi yang dibuat serta analisis dan pembahasan dari pengujian tersebut. Pengujian dilakukan untuk mengetahui keberhasilan aplikasi yang dibuat untuk menyelesaikan permasalahan yang di definisikan pada rumusan masalah.

Bab lima berisikan kesimpulan terhadap hasil implementasi dan solusi yang dipaparkan untuk menyelesaikan permasalahan. Selain itu, terdapat bagian saran yang memaparkan saran pengembangan dan perbaikan yang dapat dilakukan untuk memperkaya fitur dan menyelesaikan permasalahan yang lebih luas.

\section{Jadwal Pelaksanaan}
\newsavebox\mybox
\begin{lrbox}{\mybox}
    \setcounter{myWeekNum}{1}
    \ganttset{%
    calendar week text={\myWeek{}}%
    }

    \begin{ganttchart}[
    vgrid={*{6}{draw=none}, dotted},
    x unit=.05cm,
    y unit title=.6cm,
    y unit chart=.6cm,
    time slot format=isodate,
    time slot format/start date=2016-09-01]{2016-09-01}{2017-04-30}
    \ganttset{bar height=.6}
    \gantttitlecalendar{year, month} \\
    \ganttbar[bar/.append style={fill=blue}]{Tahap 1}{2016-09-01}{2016-11-30}\\
    \ganttbar[bar/.append style={fill=blue}]{Tahap 2}{2016-10-01}{2016-11-15}\\
    \ganttbar[bar/.append style={fill=blue}]{Tahap 3}{2016-11-01}{2016-12-15}\\
    \ganttbar[bar/.append style={fill=blue}]{Tahap 4}{2016-12-15}{2017-03-01}\\
    \ganttbar[bar/.append style={fill=blue}]{Tahap 5}{2017-02-01}{2017-04-30}
    \end{ganttchart}
\end{lrbox}

Pengerjaan tugas akhir ini direncanakan mulai pada September 2016 sampai April 2016. Pelaksanaan tugas akhir ini dibagi menjadi 5 tahap yang dapat dipetakan kepada metodologi pengerjaan sebagai berikut,
\begin{enumerate}
    \item Tahap 1: Studi Literatur
    \item Tahap 2: Analisis Masalah
    \item Tahap 3: Perancangan Solusi
    \item Tahap 4: Implementasi
    \item Tahap 5: Pengujian dan Analisis Hasil
\end{enumerate}
Jadwal pelaksanaan tugas akhir berdasarkan metodologi pengerjaan tugas akhir dapat dilihat pada Tabel \ref{Gantt-Chart} dibawah ini.
\begin{table}[b]
\centering
\caption{Gantt Chart jadwal pelaksanaan tugas akhir}
\label{Gantt-Chart}
\tikz{
  \node[inner sep=0pt,outer sep=0pt] (gantt)
  {\begin{tabular}{c}
    \toprule
    \resizebox{\textwidth}{!}{\usebox\mybox} \\
    \bottomrule
   \end{tabular}%
   };
}   
\end{table}
    \chapter{Studi Literatur}

Pada bab ini akan mendeskripsikan kajian literatur yang terkait dengan persoalan tugas akhir. Studi literatur ini akan dijadikan dasar dalam melakukan penyelesaian persoalan.

\section{Penggunaan \textit{Spreadsheet}}
Secara harafiah, \textit{spreadsheet} adalah suatu perangkat lunak yang dapat melakukan kalkulasi terhadap angka serta mengorganisir informasi yang ada di dalamnya berdasarkan kolom dan baris \parencite{meriamwebster-spreadsheet}. \textit{Spreadsheet} dapat digunakan untuk melakukan kalkulasi terhadap suatu rumus atau formula yang sulit jika dikalkulasikan dengan cara manual. Selain itu, \textit{spreadsheet} dapat juga digunakan untuk melakukan ramalan terhadap suatu perubahan variabel masukan. Pada perkembangannya, \textit{spreadsheet} memiliki fitur-fitur tambahan seperti visualisasi data dan ekstraksi data penting dari kumpulan data yang ada.

Penelitian tentang penggunaan \textit{spreadsheet} pada bisnis pernah dilakukan sebelumnya pada tahun 2014. Subjek yang diteliti adalah akuntan manajemen \parencite{Bradbard2014}. Pada penelitian tersebut, didapatkan gambaran umum mengenai penggunaan \textit{spreadsheet} secara umum. Menurut hasil penelitian tersebut beberapa fitur yang sering digunakan oleh pengguna \textit{spreadsheet} secara terurut dari yang paling sering digunakan adalah sebagai berikut,

\begin{enumerate}
    \item Menghitung fungsi matematika dasar (tambah, kurang, kali, bagi, dan lainnya)
    \item Mengelola \textit{worksheet} dan \textit{workbook} (menambahkan, menghapus, merubah nama, dan lainnya)
    \item Melakukan perubahan format dasar (menebalkan, memberi garis bawah, format angkat, dan lainnya)
    \item Melakukan pengurutan data, penghitungan subtotal, serta meringkas data
    \item Menggunakan fitur \textit{cell addressing} baik absolut maupun relatif
    \item Penggunaan fungsi kondisi (IF, COUNTIF), fungsi logika (AND, OR), fungsi pencarian (VLOOKUP, HLOOKUP), menautkan \textit{workbook} lain, serta fungsi pembulatan (ROUND, CEILING, FLOOR)
\end{enumerate}

Penggunaan \textit{spreadsheet} sangat bergantung kepada domain bisnis atau organisasi yang menggunakan. Pada bisnis yang berorientasi komersial, \textit{spreadsheet} dapat digunakan sebagai alat bantu perhitungan laba, pengeluaran, investasi, dan pajak. Pada organisasi-organisasi non komersial, \textit{spreadsheet} dapat digunakan sebagai salah satu bentuk basis data yang menangani penyimpanan, pengelolaan, dan pengumpulan data yang mudah dan cepat.

\section{Kesalahan dalam Penggunaan \textit{Spreadsheet}}
Penelitian telah dilakukan oleh Panko \parencite{Panko1998} untuk mengetahui banyaknya kesalahan yang terjadi pada pengembangan \textit{spreadsheet} terutama pada sektor bisnis. Dari penelitian ini, didapatkan bahwa 20\% hingga 40\% \textit{spreadsheet} mengandung kesalahan. Pada kasus tertentu, bahkan ditemukan 90\% \textit{spreadsheet} yang diteliti memiliki kesalahan \parencite{Journal1996}. 

Penelitian yang dilakukan oleh Panko juga menemukan 88\% dari 113 \textit{spreadsheet} yang diaudit melalui 7 lebih studi yang diteliti. Beberapa hasil yang telah di rangkum oleh penelitian tersebut mengunakan \textit{spreadsheet} yang digunakan di dunia nyata dapat dilihat pada Tabel \ref{StudiKesalahan}.
  \begin{longtable}{ | p{3cm} | r | R{2cm} | r | r | r | l | }
    \caption{Studi terhadap Kesalahan pada \textit{Spreadsheet}}
    \label{StudiKesalahan}\\ \hline
    \centering\bfseries{Pembuat} & \bfseries{Tahun} & \centering\bfseries{Jumlah yang Diaudit} & \bfseries{Rata-rata Sel} & \bfseries{Persentase Error} \\ \hline
    \endfirsthead
    \hline
    \centering\bfseries{Pembuat} & \bfseries{Tahun} & \centering\bfseries{Jumlah yang Diaudit} & \bfseries{Rata-rata Sel} & \bfseries{Persentase Error} \\ \hline
    \endhead
    Davies \& Ikin & 1987 & 19 & - & 21\% \\ \hline
    Cragg \& King & 1992 & 20 & 50 - 10000 & 25\% \\ \hline
    Butler & 1992 & 273 & - & 11\% \\ \hline
    Dent & 1994 & Tidak diketahui & - & 30\% \\ \hline
    Hicks & 1995 & 1 & 3856 & 100\% \\ \hline
    Coopers \& Lybrand & 1997 & 23 & 150+ & 91\% \\ \hline
    KPMG & 1998 & 22 & - & 91\% \\ \hline
    Lukasic & 1998 & 2 & 2270 - 7027 & 100\% \\ \hline
    Butler & 2000 & 7 & - & 86\% \\ \hline
    Clermont, Hanin, \& Mittermeier & 2002 & 3 & - & 100\% \\ \hline
    Lawrence and Lee & 2004 & 30 & 2182 & 100\% \\ \hline
    Powell, Lawson, and Baker & 2007 & 25 & - & 64\% \\ \hline
    Powell, Baker \& Lawson & 2007 & 50 & - & 86\% \\ \hline
  \end{longtable}
Dari kumpulan data diatas, dapat dilihat bahwa didalam pembentukan \textit{spreadsheet} pada bidang bisnis, tidak mungkin terlepas dari kesalahan. Dengan tingginya tingkat kesalahan ini, bisnis dapat mengalami kerugian secara material maupun moral yang cukup besar \parencite{EUSPRIGHorrorStories}. Hal ini mengindikasikan bahwa tingginya tingkat kesalahan harus dapat diselesaikan agar tidak terjadi kerugian di dalam penggunaan \textit{spreadsheet} terutama dalam bisnis.

\section{Tipe Kesalahan dalam Penggunaan \textit{Spreadsheet}}
Tingkat fleksibilitas \textit{spreadsheet} yang tinggi memberikan keleluasaan kepada penggunanya untuk melakukan banyak manipulasi dan pengelolaan data. Tingginya fleksibilitas ini dapat berakibat mudahnya \textit{human error} terjadi pada saat penggunaan \textit{spreadsheet} yang menyebabkan terjadinya kesalahan-kesalahan pada data. Tipe-tipe kesalahan pada \textit{spreadsheet} dapat dibagi menjadi dua jenis tipe kesalahan yakni kesalahan kuantitatif, dan kesalahan kualitatif \parencite{Panko1998}. 

    \subsection{Kesalahan Kualitatif}
    Kesalahan kualitatif merupakan kesalahan yang berhubungan dengan kualitas \textit{spreadsheet} tersebut. Beberapa kesalahan yang dapat diklasifikasikan sebagai kesalahan kualitatif adalah \parencite{Powell2009}:

    \begin{enumerate}
        \item Melakukan \textit{hard-code} pada suatu angka di dalam formula
        \item Menggunakan formula yang panjang dalam perhitungan
        \item Susunan data yang tidak direncanakan dengan baik
        \item Tidak adanya dokumentasi mengenai \textit{spreadsheet} yang dibuat
    \end{enumerate}

    Kesalahan ini tidak langsung mengakibatkan nilai hasil keluaran yang salah namun menurunkan kualitas dari \textit{spreadsheet} tersebut \parencite{Rajalingham2001}. Selain itu, kesalahan kualitatif dapat menyebabkan kesalahan kuantitatif terutama pada saat penggunaan fungsi analisis \textit{what-if} pada \textit{spreadsheet} \parencite{Panko1998}.

    \subsection{Kesalahan Kuantitatif}
    Kesalahan ini mengakibatkan \textit{spreadsheet} mengeluarkan hasil dan nilai yang salah didalam operasi perhitungannya. Kesalahan kuantitatif dapat dibagi menjadi tiga tipe kesalahan yakni \parencite{Panko1998}:

    \begin{enumerate}
        \item Kesalahan mekanikal (\textit{mechanical error}) yang biasanya terjadi akibat kesalahan pengetikan angka atau rujukan sel yang salah pada suatu formula
        \item Kesalahan logika (\textit{logical error}) yang terjadi pada pembuatan formula yang salah atau penggunaan fungsi yang tidak tepat
        \item Kesalahan akibat kelalaian pada interpretasi situasi atau spesifikasi yang diberikan sehingga \textit{spreadsheet} yang dihasilkan tidak sesuai dengan domain permasalahan yang ada \parencite{Powell2009} (\textit{ommision error})
    \end{enumerate}

\section{Penanganan Kesalahan pada \textit{Spreadsheet}}
Berdasarkan penelitian yang dilakukan oleh Panko \parencite{Panko1998}, dijabarkan beberapa metode untuk menangani dan mengurangi kesalahan yang sering terjadi. Beberapa metode yang dapat digunakan yakni:

    \begin{enumerate}
        \item Membangun \textit{preliminary design} sebelum pembuatan \textit{spreadsheet} agar terdapat perencanaan yang baik di dalam pembangunan data di dalam \textit{spreadsheet}
        \item Melakukan proteksi terhadap sel yang tidak boleh diubah.
        \item Melakukan pengecekan terhadap semua rumus dan formula yang dimasukan bahkan hingga rumus yang cukup sederhana dengan cara melakukan pengecekan manual.
        \item Membuat dokumentasi untuk \textit{spreadsheet} yang dibuat.
        \item Tidak menekan pembuat \textit{spreadsheet} terhadap kesalahan yang dibuat dengan memberikan hukuman. Kesalahan yang terjadi pada \textit{spreadsheet} umumnya masih berada pada batas normal \textit{human error} sehingga memberikan hukuman akan membuat rasa takut dalam melaporkan kesalahan.
        \item Melakukan inspeksi terhadap formula, rumus, dan kode yang dibuat baik oleh individual maupun secara berkelompok.
    \end{enumerate}

\section{Struktur pada Penyimpanan \textit{Spreadsheet}}
Konsep dasar pada \textit{spreadsheet} modern adalah sebuah aplikasi yang berupa sekumpulan sel terdiri dari baris dan kolom yang disebut \textit{sheet}. Sel ini dapat diisi data berupa data mentah maupun formula. Data mentah biasanya berupa angka, teks, tanggal, dan nilai mata uang. Formula merupakan perintah yang dapat dimengerti komputer untuk menghitung dan memanipulasi data pada sel. Data hasil pengolahan dan masukan pada \textit{spreadsheet} pada dasarnya disimpan dalam bentuk sel yang namanya terdiri dari nama kolom dan nilai baris (Contoh: A1 untuk kolom pertama dan baris pertama). Masing-masing sel tersebut akan menyimpan \textit{properties} dari sel yang dapat berupa \textit{value} yang diisikan, format sel, serta format data yang digunakan. 

Pada masing-masing aplikasi, cara penyimpanannya dapat berbeda satu dengan yang lain walaupun memiliki kemiripan. Bagian ini akan dibahas struktur yang digunakan pada tiga jenis aplikasi yang memiliki tipe yang berbeda; Microsoft Excel 2007 (\textit{Offline Spreadsheet}), Open Office (\textit{Offline and Open Source Spreadsheet}), EtherCalc (\textit{Online and Open Source Spreadsheet}), Google Sheet (\textit{Online Spreadsheet}).

    \subsection{Struktur Penyimpanan pada Microsoft Excel 2007}

    \subsection{Struktur Penyimpanan pada Open Office}

    \subsection{Struktur Penyimpanan pada EtherCalc}

    \subsection{Struktur Penyimpanan pada Google Sheet}

\section{\textit{Data Governance}}


\section{Mekanisme Penyimpanan Data}
Jelasin disini alternatif yang mungkin sebagai penyimpanan data buat hasil dari inputan lewat spreadsheet
Juga jelasin kenapa harus menggunakan basisdata buat penyimpanan
mungkin bisa ditulis dan dijelaskan di data governance???? 
masih bingung buat ngehubunginnnya sih :((((


\subsection{Basis Data Relasional}
Basisdata relasional merupakan sebuah basisdata digital yang berbasiskan pada \textit{paper} yang ditulis oleh E. F. Codd pada tahun 1970. Pada basisdata relasional, model data diatur kedalam bentuk tabel atau relasi yang terdiri dari baris dan kolom, dengan sebuah kunci (\textit{key}) unik untuk setiap barisnya. Sebuah tabel mengambarkan koleksi objek atau relasi yang memiliki kesamaan jenis atau sifat dimana setiap kolomnya dapat mengambarkan suatu objek atau relasi yang biasa disebut sebagai \textit{record} \parencite{codd1970relational, OracleRDB}.

    \subsubsection{Struktur Basis Data Relasional}
    \begin{figure}[htb]
        \centering
        \includegraphics[width=0.6\textwidth]{resources/chapter-2-relational-db.png}
        \caption{Ilustrasi basisdata relasional}
        \label{IlustrasiRDB}
    \end{figure}
    E. F. Codd mendefinisikan basisdata relasional dapat dijelaskan secara sederhana menggunakan representasi tabel atau \textit{array}. Secara terminologi, tabel disebut juga sebagai relasi dan merupakan kumpulan \textit{tuple} yang memiliki atribut yang sama. Atribut direpresentasikan sebagai kolom sedangkan \textit{tuple} direpresentasikan sebagai baris. Gambar \ref{IlustrasiRDB} merupakan ilustrasi dari sebuah tabel pada basisdata relasional.

    Selain terminologi tersebut, tabel atau \textit{array} dapat dikatakan sebuah data yang relasional jika dapat memenuhi beberapa sifat berikut:

    \begin{enumerate}
        \item Setiap baris merepresentasikan sebuah \textit{tuple} sejumlah \textit{n} relasi R
        \item Keterurutan baris tidak perlu diperhatikan (\textit{immaterial})
        \item Setiap baris harus berbeda satu dengan yang lainnya
        \item Keterurutan kolom perlu diperhatikan sesuai dengan domain yang dimodelkan
        \item Setiap kolom diberikan label yang sesuai dengan nama domain yang dimodelkan
    \end{enumerate}

    Sebagai contoh, pada Tabel \ref{ContohTabel} dapat dilihat merupakan contoh sebuah tabel bernama \textit{mahasiswa} dengan 3 derajat relasi yang memiliki atribut NIM, nama, dan jurusan seorang mahasiswa. Setiap baris merupakan kombinasi data yang unik dan tidak berulang.

    \begin{table}[htb]
        \caption{Sebuah relasi pada domain mahasiswa}
        \label{ContohTabel}
        \begin{center}
            \begin{tabular}{ l c c c }
                \hline
                mahasiswa & NIM & nama & jurusan \\
                & 13513042 & Feryandi N. & 135 \\
                & 13613006 & Dani Y. P. & 134 \\
                & 18013024 & Haidar A. D. & 180 \\
                \hline
            \end{tabular}
        \end{center}
    \end{table}

    Konsep lain yang juga dijelaskan oleh Codd adalah adanya \textit{primary key} dan \textit{foreign key}. \textit{Primary key} adalah sebuah nilai unik yang dimiliki oleh sebuah \textit{tuple}, sehingga \textit{tuple} tersebut dapat diidentifikasi secara unik hanya dengan menggunakan nilai tersebut. Pada pemodelan data, relasi satu dengan yang lainnya dapat saling berhubungan hal dapat direpresentasikan menggunakan \textit{foreign key}. Sebuah nilai dapat dikatakan \textit{foreign key} jika nilai tersebut bukanlah merupakan \textit{primary key} pada relasi R, tetapi merupakan \textit{primary key} pada relasi S lainnya \parencite{codd1970relational}. Pada Tabel \ref{ContohTabel}, NIM merupakan \textit{primary key} dari relasi tersebut dan jurusan merupakan \textit{foreign key} yang merujuk ke \textit{primary key} pada tabel relasi lain.

    \subsubsection{\textit{Relational Database Management System (RDBMS)}}

    \subsubsection{Sifat Transaksi pada RDBMS}    

\subsection{Basis Data Non-Relational (NoSQL)}
    
\section{Studi Terkait / Penelitian Terkait}
\blindtext

    \chapter{Analisis Masalah Penanganan Kesalahan Pada \textit{Spreadsheet}}

Pada bab ini akan diuraikan analisis persoalan penanganan kesalahan pada \textit{spreadsheet} yang telah diuraikan pada Bab I. Hasil dari bab ini digunakan untuk merancang aplikasi yang akan diimplementasikan seperti yang dijelaskan pada Bab IV.

\section{Aspek Penanganan Kesalahan Pada \textit{Spreadsheet}} \label{AspekAplikasi}
Pada Subbab \ref{KesalahanPenggunaan} dijelaskan bahwa terdapat dua jenis kesalahan yang dapat terjadi pada penggunaan \textit{spreadsheet} yakni kesalahan kualitatif dan kesalahan kuantitatif. Kesalahan kualitatif merupakan kesalahan penggunaan \textit{spreadsheet} yang berhubungan dengan kualitas dan prosedur penggunaan. Contoh kesalahan kualitatif yang sering terjadi adalah penggunaan \textit{spreadsheet} sebagai basis data. Kesalahan kuantitatif merupakan kesalahan yang menyebabkan keluaran menjadi salah. Contoh kesalahan kuantitatif yang sering terjadi adalah kesalahan masukan yang tidak divalidasi.

Pencegahan untuk kesalahan-kesalahan ini dapat dilihat pada Subbab \ref{PenangananKesalahan}. Salah satu caranya adalah pembuatan \textit{preliminary design} terhadap \textit{spreadsheet} yang dibuat. Hal ini dapat dilakukan dengan cara pengembangan aplikasi pengumpulan data berbentuk \textit{spreadsheet} yang menangani kolaborasi, validasi, dan penyimpanan. Dengan aplikasi ini diharapkan dapat menangani kesalahan kualitatif seperti yang dijelaskan pada contoh sebelumnya yakni penggunaan \textit{spreadsheet} sebagai basis data dengan cara menghubungkan \textit{spreadsheet} ke basis data yang sesungguhnya secara langsung. Serta menangani kesalahan kuantitatif dengan cara melakukan validasi masukan.

Aplikasi \textit{spreadsheet} yang baik harus dapat diakses dan digunakan secara kolaboratif yang dapat dilakukan dengan membuat aplikasi \textit{spreadsheet} tersebut berbasis \textit{web}. Aspek-aspek aplikasi \textit{web} yang perlu diperhatikan dalam membangun aplikasi \textit{spreadsheet} adalah:
\begin{enumerate}
	\item \textit{Response time}\\
	Membangun aplikasi berbasis \textit{web} menggunakan HTTP sebagai protokol komunikasi dan tentunya didalam komunikasi tersebut membutuhkan waktu untuk mengirimkan \textit{request} dari klien. Jika waktu yang dibutuhkan untuk melakukan respon lambat, akan sulit terjadi kolaborasi dan memperbesar kemungkinan terjadinya \textit{race condition} pada masukan pengguna. Sehingga diperlukan waktu respon yang cepat untuk dapat menangani banyak \textit{request} dalam satu waktu dan dalam waktu yang singkat.

	\item Akses yang konkuren\\
	Aplikasi berbasis \textit{web} yang akan dibuat harus dapat diakses secara konkuren yakni diakses bersama-sama oleh banyak klien dalam satu waktu. Pengaksesan secara konkuren dapat berdampak pada terpanggilnya banyak \textit{query} dalam satu waktu. Oleh karena itu aplikasi yang dibuat harus dapat menjalankan secara konkuren setiap \textit{request} yang masuk.

	\item Akses basis data\\
	Akses terhadap basis data dibutuhkan sebagai media penyimpanan yang persisten dan konsisten. Oleh karena itu, akses terhadap basis data diperlukan untuk kemudahan penyimpanan dan pengambilan data serta mengatasi permasahalan ketidaksamaan \textit{versioning} diantara suatu \textit{file spreadsheet}.

\end{enumerate}

Keempat aspek pada aplikasi berbasis \textit{web} tersebut akan menjadi aspek utama yang dikembangkan didalam pembuatan aplikasi \textit{spreadsheet} berbasis aplikasi \textit{web}.

Berdasarkan pada Subbab \ref{KesalahanPenggunaan}, terdapat dua jenis tipe kesalahan \citep{Panko1998} yang bisa terjadi pada penggunaan \textit{spreadsheet} yakni:
\begin{enumerate}
	\item Kesalahan Kualitatif

	Kesalahan kualitatif merupakan kesalahan yang berhubungan dengan kualitas yang dipengaruhi oleh prosedur pembuatan dan rancangan pada \textit{spreadsheet} tersebut. Contoh kesalahan kualitatif yang sering terjadi adalah penggunan \textit{spreadsheet} sebagai media penyimpanan atau basis data. Hal ini tidak sesuai dengan kegunaan utama dari \textit{spreadsheet} sebagai media kalkulasi dan perhitungan statistik.

	\item Kesalahan Kuantitatif

	Kesalahan kuantitatif merupakan kesalahan yang menyebabkan hasil perhitungan menjadi salah atau tidak valid. Contoh kesalahan jenis ini yang sering terjadi adalah kesalahan mekanikal dimana pengguna salah memasukan data yang tidak sesuai dengan konstrain yang ada. Hal ini dapat diatasi dengan adanya validasi sebelum data diterima. 
\end{enumerate}

Dari kedua tipe tersebut, pada Tugas Akhir ini akan lebih ditekankan kepada kesalahan kuantitatif yang bertipe kesalahan mekanikal dengan memanfaatkan teknik validasi pada masukan. 

Dengan menggunakan aplikasi ini, terdapat perubahan alur penggunaan \textit{spreadsheet}. Gambar dibawah ini merupakan alur penggunaan aplikasi \textit{spreadsheet} biasa pada Gambar \ref{GambarWorkflowBiasa} dan aplikasi \textit{spreadsheet} yang dibangun pada Tugas Akhir ini pada Gambar \ref{GambarWorkflow}. 

	\begin{figure}[htb]
	    \centering
	    \includegraphics[width=0.7\textwidth]{resources/chapter-3-workflow-biasa.jpg}
	    \caption{Alur Kerja Aplikasi \textit{Spreadsheet} Biasa}
		\label{GambarWorkflowBiasa}
	\end{figure}

	\begin{figure}[htb]
	    \centering
	    \includegraphics[width=1\textwidth]{resources/chapter-3-workflow.jpg}
	    \caption{Alur Kerja Aplikasi \textit{Spreadsheet} yang Akan Dibuat}
		\label{GambarWorkflow}
	\end{figure}

Dapat dilihat bahwa terdapat proses tambahan yang dilakukan pada aplikasi yang akan dibuat. Proses tersebut terdiri dari pencarian bagian label dan data serta proses validasi. Hasil dari identifikasi label dan data akan dijadikan \textit{tuple} relasional yang disimpan dalam bentuk basis data. 

\section{Isu Penting pada Aplikasi \textit{Spreadsheet}}
Di dalam membangun aplikasi \textit{spreadsheet} tersebut, terdapat beberapa isu yang harus ditangani yakni:
\begin{enumerate}
	\item Pemilihan model interaksi aplikasi \textit{spreadsheet} dimana pengguna dapat membentuk struktur tempat pengguna lain memasukan input. Contohnya dalam bentuk tabel atau formulir. Pengguna juga melengkapi domain dan batasan data yang dimasukkan.
	\item Setelah pengguna selesai merancang struktur masukan, diperlukan penentuan bagian label dan data yang berkaitan dengan label tersebut. Bagian label dan data akan dijadikan menjadi bentuk relasional yang dapat disimpan dalam basis data
	\item Pengguna lain yang hanya memiliki kapabilitas pengisian data, hanya dapat mengisi data sesuai struktur yang diberikan. Sehingga perlu melakukan validasi data dan dicek kesesuaiannya sesuai dengan batasan yang diberikan sehingga memenuhi domain yang ditentukan pada basis data.
	\item Setelah proses validasi terpenuhi, diperlukan cara penyimpanan data sesuai dengan relasi \textit{tuple} yang telah ditentukan. Pada saat pembukaan \textit{file spreadsheet} tersebut, akan dilakukan pemulihan data yang telah disimpan pada basis data untuk ditampilkan kembali.
\end{enumerate}

\section{Model Interaksi Pengguna}
Di dalam pembangunan perangkat lunak \textit{spreadsheet} untuk mengurangi kesalahan, dapat diidentifikasikan dua model interaksi yang dapat diimplementasi. Model interaksi yang pertama adalah menggunakan formulir sebagai basis masukan data dan model yang kedua adalah menggunakan aplikasi \textit{spreadsheet} secara langsung sebagai media input data.
	\subsection{Berbasis Formulir}
	Model interaksi ini menggunakan \textit{spreadsheet} sebagai tempat perancangan formulir. Pembuat formulir akan menentukan area label dan input secara manual serta diidentifikasikan berdasarkan warna atau properti lain yang unik pada sel tersebut. Formulir akan dibangkitkan oleh aplikasi agar menjadi bentuk HTML dan terhubung ke basis data. Pengisian data oleh pengguna dilakukan melalui formulir yang dibangkitkan dan dapat diakses melalui web. Beberapa cara dapat dilakukan untuk mengimplementasikan teknik ini antara lain, mengembangkan dari aplikasi \textit{spreadsheet} yang telah ada menggunakan \textit{plugin} atau membuat aplikasi baru yang dapat melakukan konversi otomatis menjadi formulir.

	\subsection{Berbasis \textit{Spreadsheet}}
	Model interaksi berbasiskan \textit{spreadsheet} menggunakan antarmuka yang telah disediakan oleh aplikasi. Penggunaannya dilakukan dengan membuat format pengisian seperti membuat tabel pada \textit{spreadsheet} pada umumnya. Pada tabel harus terdapat label dan data sehingga metadata minimal yang dibutuhkan dapat dicapai. Fitur-fitur yang ada pada \textit{spreadsheet} juga tetap dapat digunakan saat pembuatan tabel yang diinginkan. Dari pembuatan tabel tersebut dilakukan identifikasi label dari data tersebut untuk selanjutnya diproses melalui penyaringan masukan dan dimasukan ke tabel yang sesuai yang ada di basis data. Untuk mengimplementasikan teknik ini harus mengubah kode pada program \textit{spreadsheet} atau mengekstensi fitur yang ada menggunakan \textit{plugin}. 

	\subsection{Perbandingan Model Interaksi}
	Kedua model interaksi tersebut memiliki beberapa perbedaan dan efek terhadap penggunaannya baik bagi sistem maupun pengguna. Pada Tabel \ref{ModelInteraksi} dijabarkan perbandingan antara kedua model interaksi tersebut.

	\begin{longtable}{ | p{3cm} | p{4cm} | p{4cm} | }
	    \caption{Perbandingan Model Interaksi}
	    \label{ModelInteraksi}\\ \hline
	    \centering\bfseries{Parameter} & \centering\bfseries{Berbasis Formulir} & \centering\bfseries{Berbasis \textit{Spreadsheet}} \tabularnewline \hline
	    \endfirsthead
	    \hline
	    \centering\bfseries{Parameter} & \centering\bfseries{Berbasis Formulir} & \centering\bfseries{Berbasis \textit{Spreadsheet}} \tabularnewline \hline
	    \endhead
	    Fungsionalitas & Berhasil memisahkan bagian operasional dan data. Pengguna hanya memodifikasi dan melakukan input pada bagian data. Bagian operasional hanya dapat dimodifikasi oleh pembuat formulir tersebut. & Jika tidak menggunakan proteksi terhadap sel yang dapat ditulis, tidak terjadi pemisahan antara data dan operasional sehingga beberapa kesalahan yang terjadi pada saat input masih dapat terjadi. \\ \hline
	    Teknologi & Dibutuhkan adanya algoritma tambahan untuk menangani formulir dan masukannya, serta melakukan konversi dan \textit{parsing} dari sel pada \textit{spreadsheet} ke dalam bentuk formulir. & Dibutuhkan algoritma dan logika \textit{parsing} yang lebih detil dan kompleks dalam menangani kasus-kasus table tertentu. \\ \hline
	    Antarmuka & Menggunakan antarmuka formulir yang kaku terurut dari atas ke bawah serta tidak dapat melihat hasil masukan secara langsung. & Struktur tabel atau masukan dapat disesuaikan dengan kebutuhan dan tidak perlu mempelajari hal lain jika sebelumnya telah menggunakan \textit{spreadsheet} sebagai media untuk memasukan data. \\ \hline
  	\end{longtable}

  	Pada Tugas Akhir ini, akan dipilih model interaksi dengan berbasis \textit{spreadsheet} sehingga pengguna dapat dengan lebih mudah didalam menggunakannya karena tidak memerlukan pembelajaran kembali di dalam menggunakan aplikasi. Selain itu, data yang dimasukan lebih mudah untuk dilihat secara menyeluruh dibandingkan dengan berbasis formulir yang hanya dapat menerima satu masukan dalam suatu formulir.

\section{Penentuan Bagian Data dan Label}
Pada pembuatan \textit{spreadsheet} pada umumnya, didapatkan bahwa kebanyakan \textit{spreadsheet} pada umumnya berbentuk tabel yang terdiri dari dua unsur utama yakni data dan label atau disebut juga tipe \textit{data frame} \citep{Chen2013}. Bagian data merupakan bagian yang biasanya dinamis dan merupakan masukan pengguna. Bagian label merupakan bagian yang memberikan keterangan dan konteks mengenai data yang dimaksud. Pada Gambar \ref{DataFrameSederhana} dapat dilihat bahwa area dengan nomor 1 disebut sebagai label yang menjelaskan data-data dibawahnya yakni pada area nomor 2.

\begin{figure}[htb]
    \centering
    \includegraphics[width=0.3\textwidth]{resources/chapter-3-simple-dataframe.png}
    \caption{Contoh \textit{Data Frame} Sederhana}
	\label{DataFrameSederhana}
\end{figure}

	\subsection{Secara Manual}
	Penentuan label dan data dilakukan oleh pengguna secara langsung saat pembuatan tabel. Pengguna sendiri yang menentukan area mana yang merupakan label dan data mana yang dijelaskan oleh label tersebut. Dengan metode manual ini, pengguna dapat menyesuaikan bentuk tabel sesuai keinginan mereka. Metode ini menyerahkan sepenuhnya tanggungjawab keterhubungan sel label dan sel data kepada pengguna.

	\subsection{Secara Otomatis}
	Pencarian label dan data dapat menggunakan algoritma seperti yang telah dijelaskan pada Subbab \ref{metodepencarian}. Mekanisme untuk mengidentifikasi label dan data dapat dilakukan melalui 3 tahapan yakni, \textit{frame finder}, \textit{hierarchy extractor}, dan \textit{tuple builder}. Pada tahap pertama yakni \textit{frame finder} bertujuan untuk mengidentifikasi wilayah nilai (data) dan wilayah atribut (label) yang dapat berupa \textit{left attribute} maupun \textit{top attribute}. Tahap selanjutnya adalah \textit{hierarchy extractor} bertujuan untuk mendapatkan hirarki dari atribut-atribut yang ada sehingga label yang tertulis dapat dicari keterhubungan dan konteksnya terhadap data yang ada. Tahap terakhir adalah \textit{tuple builder} yang mentrasformasikan data dan label tersebut dalam bentuk \textit{tuple} yang dapat diterima oleh basis data relasional.

	\subsection{Perbandingan Metode Penentuan}
	Untuk mengetahui perbandingan kedua metode, pada Tabel \ref{MetodePenentuan} dijelaskan kelebihan dan kekurangan dua metode yang telah disebutkan sebelumnya.
	\begin{longtable}{ | p{3cm} | p{4cm} | p{4cm} | }
	    \caption{Perbandingan Metode Penentuan Data dan Label}
	    \label{MetodePenentuan}\\ \hline
	    \centering\bfseries{Keterangan} & \centering\bfseries{Manual oleh Pengguna} & \centering\bfseries{Otomatis} \tabularnewline \hline
	    \endfirsthead
	    \hline
	    \centering\bfseries{Keterangan} & \centering\bfseries{Manual oleh Pengguna} & \centering\bfseries{Otomatis} \tabularnewline \hline
	    \endhead
	    Kelebihan & Tingkat akurasi lebih tinggi karena data dan label yang ditentukan sesuai dengan keinginan pengguna. & Kenyamanan dalam penggunaan karena pengguna tidak perlu melakukan interferensi tambahan. Selain itu, sistem kemungkinan data dan label yang diambil dapat diubah ke dalam bentuk \textit{tuple} relasional lebih tinggi. \\ \hline
	    Kekurangan & Interferensi pengguna yang banyak dan mungkinnya data dan label tidak dapat dijadikan bentuk \textit{tuple} relasional. & Algoritma ini hanya optimal jika digunakan pada tabel yang terurut vertikal. \\ \hline
  	\end{longtable}
  	Dari perbandingan diatas, dapat dilihat terdapat kekurangan dan kelebihan didalam menggunakan salah satu metode tersebut. Berdasarkan hal tersebut, yang akan dipilih sebagai metode penentuan data dan label adalah gabungan dari kedua metode tersebut. Sistem awalnya akan melakukan \textit{parsing} terhadap struktur yang telah dibuat oleh pengguna, kemudian pengguna dapat melihat hasil dari \textit{parsing} tersebut sehingga pengguna dapat memperbaiki jika terdapat hasil pelabelan yang salah. Dengan metode gabungan ini diharapkan dapat meningkatkan akurasi dan mengurangi kesalahan \textit{parsing} yang terjadi namun tetap memberikan kenyamanan terhadap pengguna serta menjaga agar hasil pelabelan tetap dapat diubah ke dalam bentuk \textit{tuple} relasional.

\section{Validasi Data}
Setelah pengguna memasukan datanya kedalam struktur yang telah ditetapkan, pengecekan data dilakukan. Tujuan dari mekanisme ini adalah untuk menghindari kesalahan masukan yang terjadi dan menyesuaikan tipe yang diterima oleh tabel pada basis data. Terdapat tiga hal utama yang divalidasi pada aplikasi ini, yakni:
\begin{enumerate}
	\item Validasi tipe data. Contohnya, data masukan yang seharusnya berupa \textit{integer} menerima masukan berupa \textit{string}.
	\item Validasi domain masukan. Pengguna dapat juga mengatur rentang \textit{integer} yang boleh dijadikan masukan atau \textit{string} apa saja yang dapat diterima oleh sistem.
	\item Validasi relasi data. Data masukan dapat berupa nilai yang harus ada pada tabel lain, contohnya terdapat 2 tabel yakni nilai mahasiswa dan identitas mahasiswa. Tabel nilai mahasiswa memiliki kolom NIM yang nilainya harus ada pada tabel identitas mahasiswa. Pada \textit{spreadsheet} biasanya hal ini diatasi dengan menggunakan perintah VLOOKUP.
\end{enumerate}
Sistem akan melakukan validasi terhadap ketiga hal tersebut dan menolak data masukan sehingga pengguna dapat memperbaiki data. 

\section{Penyimpanan dan Pemulihan Data}
Data yang telah dimasukan oleh pengguna baik data bentuk struktur \textit{spreadsheet} maupun data masukan pada struktur disimpan ke dalam basis data yang presisten. Terdapat 2 jenis data yang harus disimpan ke dalam basis data di dalam membangun aplikasi \textit{spreadsheet} ini, yakni:

\begin{enumerate}
	\item Penyimpanan \textit{File Spreadsheet} \\
	\textit{File spreadsheet} yang dimaksud adalah data struktural seperti \textit{value} pada suatu sel, \textit{properties} suatu sel seperti warna, \textit{border}, dan \textit{alignment}, serta hal-hal lain yang berhubungan dengan bagaimana suatu \textit{file spreadsheet} ditampilkan pada aplikasi. Tipe penyimpanan yang dapat digunakan untuk menampung data ini adalah NoSQL karena tipe ini cocok untuk menangani data yang kurang terstruktur seperti \textit{file spreadsheet} yang struktur penempatan datanya berbeda tiap pengguna. Cara penyimpanan yang cocok untuk menangani struktur ini adalah \textit{document based} atau \textit{key-value}.

	\item Penyimpanan Data dan Label \\
	Data yang disimpan merupakan hasil dari pendeteksian label dan data yang akan diubah menjadi \textit{tuple} relasional yang dapat diterima oleh basis data relasional. Contoh pengubahan yang terjadi dapat dilihat pada Gambar \ref{RelationalTuple}.

	\begin{figure}[htb]
	    \centering
	    \includegraphics[width=0.6\textwidth]{resources/chapter-3-relational-tuple.png}
	    \caption{Contoh \textit{Tuple} Relasional}
		\label{RelationalTuple}
	\end{figure}

	Setelah data dan label dijadikan \textit{tuple} relasional, \textit{tuple} dimasukan ke dalam basis data dengan menggunakan operasi \textit{insert} ke dalam tabel yang terhubung dengan data tersebut. Pada saat pemulihan data, data yang disimpan akan kembali ditampilkan pada sel yang bersesuaian sehingga pengguna dapat melanjutkan perubahan pada \textit{spreadsheet} dan berkolaborasi. Tipe penyimpanan yang cocok digunakan adalah basis data relasional (SQL) karena data yang dibuat dalam bentuk tabel pada umumnya dapat dikonversikan menjadi \textit{tuple} relasional.
\end{enumerate}

\section{Rencana Tindak Lanjut}
Berdasarkan analisa yang telah dijelaskan pada bab-bab sebelumnya, pada Tugas Akhir ini akan dibangun aplikasi \textit{spreadsheet} dengan menggunakan EtherCalc sebagai teknologi utama. Pada Subbab \ref{TeknologiSpreadsheet} telah dijelaskan bahwa EtherCalc merupakan perangkat lunak \textit{spreadsheet} yang \textit{open source} dan memiliki kemampuan kolaborasi didalam penggunaannya. Sehingga dengan menggunakan EtherCalc, aspek-aspek yang telah dijelaskan pada Subbab \ref{AspekAplikasi} yang harus ada didalam pembangunan \textit{spreadsheet} ini sudah dapat dipenuhi. Pada Tugas Akhir ini akan dilakukan pengembangan dari EtherCalc yang berupa identifikasi label dan data, penanganan validasi data dan keterhubungannya dengan basis data relasional. Alur kerja dari aplikasi \textit{spreadsheet} yang akan dibuat dapat dilihat pada Gambar \ref{GambarWorkflow} pada Subbab \ref{AspekAplikasi}.
    \chapter{Rancangan, Implementasi, dan Pengujian}

\section{Perancangan Perangkat Lunak}
Subbab perncangan perangkat lunak menjelaskan deskripsi aplikasi, analisis kebutuhan fungsional dan non-fungsional, desain perangkat lunak, serta interaksinya.

	\subsection{Deskripsi Umum Aplikasi}
	Aplikasi pengumpulan data yang dibuat merupakan pengembangan terhadap aplikasi \textit{spreadsheet} kolaboratif yang sudah ada sebelumnya yakni EtherCalc. Aplikasi yang ditambahkan ini yang akan melakukan pengaturan koneksi ke basis data. Saat pengguna menginginkan penyimpanan data ke dalam basis data yang dituju, aplikasi ini akan melakukan pencarian bagian label dan data dan melakukan validasi masukan sebelum memasukkan data ke dalam basis data. 

	\subsection{Spesifikasi Kebutuhan}
	Pada subbab ini akan dipaparkan \textit{use case} aplikasi yang akan dibuat serta kebutuhan fungsional dan non-fungsional dari aplikasi. Kasus penggunaan oleh pengguna diberi ID dengan format UC-XX dengan UC menyatakan \textit{use case} dan XX menyatakan nomor. Pengguna adalah pihak yang menggunakan aplikasi \textit{spreadsheet} yang sudah ditambahkan fitur pengumpulan data. Kasus penggunaan oleh pengguna dijelaskan pada Tabel \ref{KebutuhanPengguna}.

	\begin{longtable}{ | p{2cm} | p{10cm} | }
	    \caption{Kasus Penggunaan oleh Pengguna}
	    \label{KebutuhanPengguna}\\ \hline
	    \centering\bfseries{ID} & \centering\bfseries{Keterangan} \tabularnewline \hline
	    \endfirsthead
	    \hline
	    \centering\bfseries{ID} & \centering\bfseries{Keterangan} \tabularnewline \hline
	    \endhead
	    UC-01 & Pengguna dapat menentukan basis data tujuan dengan konfigurasi basis data yang diinginkan. \\ \hline
	    UC-02 & Pengguna dapat memuat \textit{spreadsheet} serta menyimpan data ke dalam basis data saat dibutuhkan. \\ \hline
	    UC-03 & Pengguna dapat memperbaiki atribut yang terdeteksi secara otomatis. \\ \hline
	    UC-04 & Pengguna dapat memberikan batasan dan validasi pada suatu domain data. \\ \hline
	\end{longtable}

	Berdasarkan kasus penggunaan di atas, dirancang kebutuhan fungsional perangkat lunak yang diberi ID dengan format FR-XX dengan FR merupakan singkatan dari \textit{functional requirement} dan XX menyatakan nomor kebutuhan. Kebutuhan fungsional dijelaskan pada Tabel \ref{KebutuhanFungsional}.

	\begin{longtable}{ | p{2cm} | p{6cm} | p{4cm} | }
	    \caption{Kebutuhan Fungsional Aplikasi}
	    \label{KebutuhanFungsional}\\ \hline
	    \centering\bfseries{ID} & \centering\bfseries{Keterangan} & \centering\bfseries{ID Use Case Terkait} \tabularnewline \hline
	    \endfirsthead
	    \hline
	    \centering\bfseries{ID} & \centering\bfseries{Keterangan} & \centering\bfseries{ID Use Case Terkait} \tabularnewline \hline
	    \endhead
	    FR-01 & Aplikasi dapat melakukan koneksi kepada basis data yang ditentukan oleh pengguna melalui data masukan berupa \textit{host}, \textit{port}, \textit{username}, \textit{password}, dan \textit{database} dari basis data yang dituju. & UC-01 \\ \hline
	    FR-02 & Aplikasi dapat melakukan perintah basis data kepada basis data yang dituju seperti CREATE, ALTER, UPDATE, DELETE, INSERT, SELECT dan DROP. & UC-01, UC-02 \\ \hline
	    FR-03 & Aplikasi menyediakan tombol untuk melakukan \textit{commit} terhadap data yang akan disimpan. & UC-02 \\ \hline
	    FR-04 & Aplikasi dapat menampilkan hasil identifikasi label dan data & UC-03 \\ \hline
	    FR-05 & Aplikasi menyediakan fitur bagi pengguna agar dapat mengubah hasil identifikasi label dan data & UC-03 \\ \hline
	    FR-06 & Aplikasi menyediakan fitur bagi pengguna agar dapat menambahkan batasan masukan pada suatu data & UC-04 \\ \hline
	    FR-07 & Aplikasi dapat melakukan validasi data masukan sesuai dengan batasan yang diberikan oleh pengguna & UC-04 \\ \hline
	\end{longtable}

	Selain kebutuhan fungsional, dijabarkan juga kebutuhan non-fungsional yang memiliki ID dengan format NF-XX dengan NF merupakan singkatan dari \textit{non-functional requrirement} dan XX menyatakan nomor. Kebutuhan non-fungsional disajikan pada Tabel \ref{KebutuhanNonfungsional}.

	\begin{longtable}{ | p{2cm} | p{6cm} | p{4cm} | }
	    \caption{Kebutuhan Non-fungsional Aplikasi}
	    \label{KebutuhanNonfungsional}\\ \hline
	    \centering\bfseries{ID} & \centering\bfseries{Keterangan} & \centering\bfseries{ID Use Case Terkait} \tabularnewline \hline
	    \endfirsthead
	    \hline
	    \centering\bfseries{ID} & \centering\bfseries{Keterangan} & \centering\bfseries{ID Use Case Terkait} \tabularnewline \hline
	    \endhead
	    NF-01 & Data masukan pengguna disimpan secara persisten. & UC-01, UC-02 \\ \hline
	    NF-02 & Aplikasi dapat berjalan diatas aplikasi \textit{spreadsheet} EtherCalc & - \\ \hline
	\end{longtable}

	\subsection{Kebutuhan Modul} \label{KebutuhanModul}
	Pembangunan fitur ini diatas aplikasi EtherCalc terdiri dari lima buah modul, yaitu:
	\begin{enumerate}
		\item Modul \texttt{player}, bertugas sebagai jembatan antara \textit{front-end} dan \textit{back-end} dari fitur.
		\item Modul \texttt{db}, bertugas untuk antarmuka baca tulis basis data.
		\item Modul \texttt{framefinder}, bertugas untuk mendeteksi secara otomatis bagian label dan data pada tabel.
		\item Modul \texttt{hierarchyfinder}, bertugas untuk mendeteksi secara otomatis tabel-tabel yang ada dalam suatu \textit{sheet}.
		\item Modul \texttt{checker}, bertugas untuk melakukan pengecekan data sebelum diteruskan ke basis data.
	\end{enumerate}

	Ketergantungan antar modul dapat dilihat pada Gambar \ref{ModuleDependency}

	\begin{figure}[htb]
	    \centering
	    \includegraphics[width=0.6\textwidth]{resources/chapter-4-module-dependecy.png}
	    \caption{Ketergantungan Antar Modul}
		\label{ModuleDependency}
	\end{figure}

	\subsection{Kolaborasi Antar Modul}
	Proses fitur ini akan dilakukan melalui modul \texttt{player} yang dapat menerima perintah pengguna melalui \textit{front-end}. Selanjutnya modul \texttt{framefinder} akan melakukan pendeteksian label dan data secara otomatis pada masing-masing tabel yang terdapat pada \textit{sheet}. Tabel-tabel tersebut didapatkan melalui modul \texttt{hierarchyfinder}. Selanjutnya, pada saat menerima perintah penyimpanan, modul \texttt{checker} akan dipanggil oleh \texttt{player}. Jika data masukan sudah benar, maka modul \texttt{db} akan melakukan pennyimpanan ke dalam basis data. Kolaborasi antar modul disajikan pada Gambar \ref{ModuleFlow}.

	\begin{figure}[htb]
	    \centering
	    \includegraphics[width=0.4\textwidth]{resources/chapter-4-module-flow.png}
	    \caption{Kolaborasi Antar Modul}
		\label{ModuleFlow}
	\end{figure}


\section{Implementasi}
Implementasi dilakukan dengan membangun modul yang telah dijabarkan pada Subbab \ref{KebutuhanModul} dengan menggunakan bahasa Javascript, menyesuaikan dengan modul lain yang telah ada pada aplikasi EtherCalc.
	\subsection{Modul Player}
	Modul \texttt{player} merupakan modul yang menjembatani masukan pengguna dari yang berasal dari \textit{front-end} sehingga dapat diterima oleh modul yang berada di \textit{back-end}. Modul ini terdiri dari fungsi-fungsi yang dapat dipanggil oleh \textit{front-end} diantaranya:
	\begin{enumerate}
		\item Save, melakukan pemanggilan terhadap modul \texttt{checker} dan melakukan penyimpanan ke basis data.
		\item Scan, melakukan identifikasi tabel melalui pemanggilan modul \texttt{framefinder} yang selanjutnya akan menampilkan hasil identifikasi dan kolom perubahan konfigurasi yang dapat diisi pengguna.
		\item SaveConfig, melakukan penyimpanan konfigurasi tabel yang dilakukan oleh pengguna.
		\item Connect, melakukan koneksi ke basis data yang dipilih.
	\end{enumerate}

	\subsection{Modul DB}
	Modul basis data digunakan sebagai antarmuka modul lain untuk melakukan operasi I/O basis data. Pada Tugas Akhir ini, basis data yang digunakan adalah MySQL.

	\subsection{Modul Framefinder}
	Modul \texttt{framefinder} melakukan pengidentifikasian terhadap tabel yang ada sehingga dapat diketahui baris yang merupakan \textit{header} dan \textit{data}. Implementasi modul ini dilakukan dengan mengikuti implementasi yang dilakukan pada penelitian yang dilakukan oleh Chen \citep{Chen2013}.
	
	\subsection{Modul Hierarchyfinder}
	Modul \texttt{hierarchyfinder} menggunakan algoritma hierarchical clustering untuk dapat mengetahui mana yang merupakan suatu kesatuan tabel pada suatu \textit{sheet}. Modul ini dapat menentukan tabel-tabel yang terdapat pada suatu \textit{sheet} yang selanjutnya akan dilakukan identifikasi label oleh modul \texttt{framefinder}.

	\subsection{Modul Checker}
	Modul \texttt{checker} memiliki tugas untuk melakukan pengecekan terhadap masukan pengguna pada konfigurasi serta melakukan pengecekan kesesuaian nilai masukan berdasarkan tipe, \textit{range}, serta relasi yang ditentukan oleh pengguna. Jika seluruh kriteria telah dipenuhi, maka modul ini akan memanggil modul \texttt{db} untuk melakukan penyimpanan data.

% \section{Pengujian}
% \blindtext

% \section{Pembahasan}
% \blindtext
    \chapter{Penutup}
Bab ini berisi hal-hal yang dapat disimpulkan dari pelaksanaan Tugas Akhir ini. Bab ini juga mencakup saran untuk pengembangan Tugas Akhir ini di masa mendatang.

\section{Kesimpulan}
Berdasarkan hasil pengembangan aplikasi pengumpulan data menggunakan \textit{spreadsheet} yang telah dilakukan. Berikut adalah kesimpulan yang diperoleh.
\begin{enumerate}
	\item Identifikasi tabel pada suatu \textit{sheet} dapat dilakukan dengan menggunakan algoritma hierarchical clustering.
	\item Identifikasi label suatu baris pada tabel dapat dilakukan dengan teknik framefinder dengan membagi label menjadi empat jenis yakni \textit{title}, \textit{data}, \textit{header}, dan \textit{footer}
	\item Penyimpanan data dari bentuk \textit{spreadsheet} ke dalam bentuk basis data dapat dilakukan dengan cepat dan diharapkan dapat mempercepat proses pengumpulan data.
\end{enumerate}

\section{Saran}
Saran yang dapat diberikan untuk pengembangan di masa mendatang adalah sebagai berikut:
\begin{enumerate}
	\item Pada pembangunan selanjutnya dapat ditambahkan fitur untuk menangani kasus tabel yang lebih rumit seperti formulir.
	\item Penambahan data pembelajaran untuk identifikasi label baris dapat dilakukan sehingga akan memperbaiki hasil identifikasi otomatis. Selain itu pada pengembangan selanjutnya dapat ditambahkan \textit{feedback} dari pengguna sebagai data pembelajaran.
\end{enumerate}
    %----------------------------------------------------------------%

    % Daftar pustaka
    % Bibliography to Daftar Pustaka
    \renewcommand{\bibname}{Daftar Pustaka}
    \cleardoublepage
    \phantomsection
    \addcontentsline{toc}{chapter}{DAFTAR PUSTAKA}
    %\printbibliography
    \bibliography{references}
    \bibliographystyle{apa}

    % Index
    \appendix

    \cleardoublepage
    \phantomsection
    %\part*{Lampiran}
    %\addcontentsline{toc}{part}{LAMPIRAN}

    %\input{chapters/appendix-1}
    %\input{chapters/appendix-2}

\end{document}
